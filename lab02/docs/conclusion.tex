\chapter*{Заключение}
\addcontentsline{toc}{chapter}{Заключение}

В результате исследования было определено, что стандартный алгоритм умножения матриц проигрывает по времени алгоритму Винограда примерно в 1.1 раза из-за того, что в алгоритме Винограда часть вычислений происходит заранее, а также сокращается часть сложных операций -- операций умножения, поэтому предпочтение следует отдавать алгоритму Винограда. 
Но лучшие показатели по времени выдает оптимизированный алгоритм Винограда -- он примерно в 1.2 раза быстрее алгоритма Винограда на размерах матриц свыше 10 из-за замены операций равно и плюс на операцию плюс-равно, а также за счёт замены операции умножения операцией сдвига, что дает проводить часть вычислений быстрее. 
Поэтому при выборе самого быстрого алгоритма предпочтение стоит отдавать оптимизированному алгоритму Винограда. 
Также стоит упомянуть, что алгоритм Винограда работает на чётных размерах матриц примерно в 1.1 раза быстрее, чем на нечётных, что связано с тем, что нужно произвести часть дополнительных вычислений для крайних строк и столбцов матриц, поэтому алгоритм Винограда лучше работает чётных размерах матриц.


В результате лабораторной работы были изучены, реализованы и исследованы алгоритмы умножения матриц -- классический алгоритм, алгоритм Винограда и оптимизированный алгоритм Винограда.

Были решены следующие задачи:

\begin{enumerate}[label=\arabic*)]
	\item изучены и реализованы алгоритмы умножения матриц - стандартный, Винограда и оптимизированный алгоритм Винограда;
    \item проведён сравнительный анализ по времени алгоритмов умножения матриц на чётных размерах матриц;
	\item проведён сравнительный анализ по времени алгоритмов умножения матриц на нечётных размерах матриц;
	\item проведён сравнительный анализ по времени алгоритмов алгоритмов между собой;
	\item подготовлен отчёт о лабораторной работе в виде расчётно-пояснительной записки к ней.
\end{enumerate}