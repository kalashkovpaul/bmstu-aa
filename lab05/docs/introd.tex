\chapter*{Введение}
\addcontentsline{toc}{chapter}{Введение}

Использование параллельной обработки открывает новые способы для
ускорения работы программ.
Конвейерная обработка является одним из примеров, где использование принципов параллельности помогает ускорить обработку данных. 
Суть та же, что и при работе реальных конвейерных лент --- материал (данное) поступает на обработку, после окончания обработки материал передается на место следующего обработчина, при этом предыдыдущий обработчик не ждёт полного цикла обработки материала, а получает новый материал и работает с ним.


\textbf{Цель работы:} изучение принципов конвейерной обработки данных.

\textbf{Задачи работы:}
\begin{enumerate}[label=\arabic*)]
	\item изучить основы конвейерной обработки данных;
	\item описать алгоритмы обработки матрицы, которые будут использоваться в текущей лабораторной работе;
	\item сравнить и проанализировать реализации алгоритмов по затраченному времени;
	\item описать и обосновать полученные результаты в отчёте о выполненной лабораторной работе, выполненном как расчётно-пояснительная записка к работе.
\end{enumerate}
