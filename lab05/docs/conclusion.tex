\chapter*{Заключение}
\addcontentsline{toc}{chapter}{Заключение}

Была достигнута цель работы: изучены принципы конвейерной обработки данных.
Также в ходе выполнения лабораторной работы были решены следующие задачи:

\begin{enumerate}[label=\arabic*)]
	\item были изучены основы конвейрной обработки данных;
	\item были описаны используемые в лабораторной работе алгоритмы обработки матрицы;
    \item были проведены сравнение и анализ трудоёмкостей алгоритмов на основе теоретических расчетов;
	\item был подготовлен отчёт о лабораторной работе, представленный как расчётно-пояснительная записка к работе.
\end{enumerate}


Исходя из полученных результатов, использование конвейрной обработки лучше линейной реализации при количестве матриц, равном 10, в 1.3 раза, а при количестве матриц, котрое равно 50, уже в 1.5 раза. Следовательно, конвейерная реализация лучше линейной при увеличении количества задач (матриц).

Также при проведении эксперимента было выявлено, что при увеличении размера матриц конвейерная реализация выдает лучшие результаты. Так, при размере матрицы в 1500 конвейерная реализация лучше в 1.4 раза, чем линейная, а при размере матриц, равном 2500, в 1.5 раза. Таким образом, следует использовать конвейрную реализацию.