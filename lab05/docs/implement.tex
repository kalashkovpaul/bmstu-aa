\chapter{Технологическая часть}

В данном разделе будут рассмотрены средства реализации, а также представлены листинги линейной и конвейрной орбаботок матрицы.


\section{Средства реализации}
В данной работе для реализации был выбран язык программирования $C++$ \cite{cpp-lang}. В текущей лабораторной работе требуется замерить время работы выполняемой программы
и визуализировать результаты при помощи графиков. Инструменты для этого присутствуют в выбранном языке программирования.

Построение графиков и функциональное тестирование было осуществлено при помощи языка программирования $Python$ \cite{python-lang}.

Время замерено с помощью \textit{std::chrono::system\_clock::now(...)} --- функции из библиотеки $chrono$ \cite{cpp-lang-chrono}.


\section{Сведения о модулях программы}
Программа состоит из пяти модулей:
\begin{enumerate}[label=\arabic*)]
	\item $main.cpp$ --- файл, содержащий точку входа;
	\item $conway$ --- модуль, содержащий код линейной и конвейерной обработок;
	\item $matrix$ --- модуль, содержаший код функций обработки матрицы;
	\item $menu$ --- модуль, содержащий алгоритм меню;
	\item $utils$ --- модуль, содержащий служебные алгоритмы. 
\end{enumerate}


\section{Реализации алгоритмов}

В листингах \ref{lst:parse_linear}--\ref{lst:stage3} представлены реализации алгоритмов линейной и конвейерной обработок матрицы, а также алгоритмы запуска трёх потоков (для нахождения среднего арифметического значений матрицы, для нахождения максимального элемента и для заполнения матрицы соответственно).

\begin{center}
    \captionsetup{justification=raggedright,singlelinecheck=off}
    \begin{lstlisting}[label=lst:parse_linear,caption=Алгоритм линейной обработки матрицы]
	void parseLinear(int count, size_t size, bool needPrinting) {
		currentTime = 0;
		std::queue<matrix_t> q1;
		std::queue<matrix_t> q2;
		std::queue<matrix_t> q3;
		
		queues_t queues = {.q1 = q1, .q2 = q2, .q3 = q3};
		
		for (int i = 0; i < count; i++) {
			matrix_t res = generateMatrix(size);
			
			queues.q1.push(res);
		}
		
		for (int i = 0; i < count; i++) {
			matrix_t matrix = queues.q1.front();
			stage1Linear(matrix, i + 1, needPrinting);
			queues.q1.pop();
			queues.q2.push(matrix);
			
			matrix = queues.q2.front();
			stage2Linear(matrix, i + 1, needPrinting);
			queues.q2.pop();
			queues.q3.push(matrix);
			
			matrix = queues.q3.front();
			stage3Linear(matrix, i + 1, needPrinting);
			queues.q3.pop();
		}
	}
	
	\end{lstlisting}
\end{center}
\clearpage

\begin{center}
\captionsetup{justification=raggedright,singlelinecheck=off}
\begin{lstlisting}[label=lst:parse_parallel,caption=Алгоритм конвейерной обработки матрицы]
void parseParallel(int count, size_t size, bool needPrinting) {
	t1.resize(count + 1);
	t2.resize(count + 1);
	t3.resize(count + 1);
	
	for (int i = 0; i < count + 1; i++) {
		t1[i] = 0;
		t2[i] = 0;
		t3[i] = 0;
	}
	
	std::queue<matrix_t> q1;
	std::queue<matrix_t> q2;
	std::queue<matrix_t> q3;
	
	queues_t queues = {.q1 = q1, .q2 = q2, .q3 = q3};
	
	
	for (int i = 0; i < count; i++) {
		matrix_t res = generateMatrix(size);
		q1.push(res);
	}
	
	std::thread threads[THREADS];
	
	threads[0] = std::thread(stage1Parallel, std::ref(q1), std::ref(q2), std::ref(q3), needPrinting);
	threads[1] = std::thread(stage2Parallel, std::ref(q1), std::ref(q2), std::ref(q3), needPrinting);
	threads[2] = std::thread(stage3Parallel, std::ref(q1), std::ref(q2), std::ref(q3), needPrinting);
	
	for (int i = 0; i < THREADS; i++) {
		threads[i].join();
	}
}
\end{lstlisting}
\end{center}
\clearpage

\begin{center}
\captionsetup{justification=raggedright,singlelinecheck=off}
\begin{lstlisting}[label=lst:stage1,caption=Алгоритм запуска 1 потока для нахождения среднего арифметического элементов матрицы]
void stage1Parallel(std::queue<matrix_t> &q1, std::queue<matrix_t> &q2, std::queue<matrix_t> &q3, bool needPrinting) {
	
	int task = 1;
	std::mutex m;
	while(!q1.empty()) {
		m.lock();
		matrix_t matrix = q1.front();
		m.unlock();
		
		logConway(matrix, task++, 1, getAverage, needPrinting);
		
		m.lock();
		q2.push(matrix);
		q1.pop();
		m.unlock();
	}
}
\end{lstlisting}
\end{center}

\begin{center}
	\captionsetup{justification=raggedright,singlelinecheck=off}
	\begin{lstlisting}[label=lst:stage2,caption=Алгоритм запуска 2 потока для нахождения максимального элемента матрицы, часть 1]
void stage2Parallel(std::queue<matrix_t> &q1, std::queue<matrix_t> &q2, std::queue<matrix_t> &q3, bool needPrinting) {
	
	int task = 1;
	std::mutex m;
	do {
		m.lock();
		bool is_q2empty = q2.empty();
		m.unlock();
		
		if (!is_q2empty) {
			m.lock();
			matrix_t matrix = q2.front();
			m.unlock();
			
			logConway(matrix, task++, 2, getMax, needPrinting);
	\end{lstlisting}
\end{center}

\begin{center}
	\captionsetup{justification=raggedright,singlelinecheck=off}
	\begin{lstlisting}[label=lst:stage2,caption=Алгоритм запуска 2 потока для нахождения максимального элемента матрицы, часть 2]
			m.lock();
			q3.push(matrix);
			q2.pop();
			m.unlock();
		}
	} while (!q1.empty() || !q2.empty());
}
	\end{lstlisting}
\end{center}

\begin{center}
	\captionsetup{justification=raggedright,singlelinecheck=off}
	\begin{lstlisting}[label=lst:stage3,caption=Алгоритм запуска 3 потока для заполнения матрицы вычисленными значениями]
void stage3Parallel(std::queue<matrix_t> &q1, std::queue<matrix_t> &q2, std::queue<matrix_t> &q3, bool needPrinting) {
	
	int task = 1;
	std::mutex m;
	
	do {
		m.lock();
		bool is_q3empty = q3.empty();
		m.unlock();
		
		if (!is_q3empty) {
			m.lock();
			matrix_t matrix = q3.front();
			m.unlock();
			
			logConway(matrix, task++, 3, fillMatrix, needPrinting);
			
			m.lock();
			q3.pop();
			m.unlock();
		}
	} while (!q1.empty() || !q2.empty() || !q3.empty());
}
	\end{lstlisting}
\end{center}
\clearpage

\section{Функциональные тесты}

В таблице \ref{tbl:functional_test} приведены тесты для функций программы. Тесты \textit{для всех функций} пройдены успешно.

\begin{table}[h]
	\begin{center}
		\begin{threeparttable}
			\captionsetup{justification=raggedright,singlelinecheck=off}
			\caption{\label{tbl:functional_test} Функциональные тесты}
			\begin{tabular}{|c|c|c|c|}
				\hline
				Алгоритм & \begin{tabular}{c}
					Кол-во \\
					 матриц 
					\end{tabular}& \begin{tabular}{c}
					Размер \\
					матриц 
				\end{tabular}& Ожидаемый результат \\
				\hline
				Конвейерная & -5 & 500 & Сообщение об ошибке \\
				\hline
				Конвейерная & 10 & -23 & Сообщение об ошибке \\
				\hline
				Линейная & 50 & 1500 & Вывод лога работы программы \\
				\hline
				Конвейерная & 10 & 100 & Вывод лога работы программы \\
				\hline
				Конвейерная & 5 & 10 & Вывод лога работы программы \\
				\hline
			\end{tabular}
		\end{threeparttable}
	\end{center}
\end{table}


\section*{Вывод}

Были представлены листинги всех алгоритмов линейной и конвейерной обработки матриц. 
Также в данном разделе была приведена информация о выбранных средствах для разработки алгоритмов и сведения о модулях программы, проведено функциональное тестирование.

