\chapter{Технологическая часть}

В данном разделе будут рассмотрены средства реализации, а также представлены листинги реализаций рассматриваемых сортировок.

\section{Средства реализации}
В данной работе для реализации был выбран язык программирования $Python$ \cite{python-lang}. В текущей лабораторной работе требуется замерить процессорное время работы выполняемой программы
и визуализировать результаты при помощи графиков. Инструменты для этого присутствуют в выбранном языке программирования.

Время работы было замерено с помощью функции \textit{process\_time(...)} из библиотеки $time$ \cite{python-lang-time}.

\section{Сведения о модулях программы}
Программа состоит из шести модулей:
\begin{enumerate}[label=\arabic*)]
	\item $main.py$ -- файл, содержащий точку входа;
	\item $menu.py$ -- файл, содержащий код меню программы;
	\item $test.py$ -- файл, содержаший код тестирования алгоритмов;
	\item $utils.py$ -- файл, содержащий служебные алгоритмы;
	\item $constants.py$ -- файл, содержаший константы программы;
	\item $algorythms.py$ -- файл, содержащий код всех алгоритмов. \newline
\end{enumerate}


\section{Реализации алгоритмов}

В листингах \ref{lst:pancake_sort}, \ref{lst:radix_sort}, \ref{lst:counting_sort}, \ref{lst:bst_sort} представлены реализации алгоритмов сортировок (блинной, поразрядной, подсчётом и бинарным деревом). \clearpage

\begin{center}
    \captionsetup{justification=raggedright,singlelinecheck=off}
    \begin{lstlisting}[label=lst:pancake_sort,caption=Алгоритм блинной сортировки]
	def pancakeSort(data, size):
		if size > 1:
			for currentSize in range(size, 1, -1):
				maxIndex = max(range(currentSize), key = data.__getitem__)
				if maxIndex + 1 != currentSize:
					if maxIndex != 0:
						data[:maxIndex + 1] = reversed(data[:maxIndex + 1])
					data[:currentSize] = reversed(data[:currentSize])
		return data
	\end{lstlisting}
\end{center}

\begin{center}
\captionsetup{justification=raggedright,singlelinecheck=off}
\begin{lstlisting}[label=lst:radix_sort,caption=Алгоритм поразрядной сортировки]
def radixSort(data, size):
	maxElement = max(data)
	place = 1
	while maxElement // place > 0:
		countingSort(data, size, place)
		place *= 10
	return data
\end{lstlisting}
\end{center}


\begin{center}
\captionsetup{justification=raggedright,singlelinecheck=off}
\begin{lstlisting}[label=lst:counting_sort,caption=Алгоритм сортировки подсчётом]
def countingSort(array, size, place):
	output = [0] * size
	count = [0] * 10
	for i in range(0, size):
		count[(array[i] // place) % 10] += 1
	for i in range(1, 10):
		count[i] += count[i - 1]
		i = size - 1
	while i >= 0:
		index = array[i] // place
		output[count[index % 10] - 1] = array[i]
		count[index % 10] -= 1
		i -= 1
	for i in range(0, size):
		array[i] = output[i]
\end{lstlisting}
\end{center}

\begin{center}
	\captionsetup{justification=raggedright,singlelinecheck=off}
	\begin{lstlisting}[label=lst:bst_sort,caption=Алгоритм сортировки бинарным деревом]
class BSTree:
	def __init__(self):
		self.root = None
	
	def __str__(self):
		rootStr = self.root.toString()[:-2]
		return "[" + rootStr + "]"
	
	def inorder(self):
		if self.root is not None:
			self.root.inorder()
	
	def add(self, key):
		new_node = BSTNode(key)
		if self.root is None:
			self.root = new_node
		else:
			self.root.insert(new_node)

def bstSort(data, n):
	bstree = BSTree()
	for x in data:
		bstree.add(x)
	return bstree
	\end{lstlisting}
\end{center}

\section{Функциональные тесты}

В таблице \ref{tbl:functional_test} приведены тесты для функций, реализующих алгоритмы сортировки. Тесты для всех сортировок пройдены \textit{успешно}.
\clearpage


\begin{table}[h]
	\begin{center}
		\begin{threeparttable}
		\captionsetup{justification=raggedleft,singlelinecheck=off}
		\caption{\label{tbl:functional_test} Функциональные тесты}
		\begin{tabular}{|c|c|c|}
			\hline
			Входной массив & Ожидаемый результат & Результат \\ 
			\hline
			$[1, 2, 3, 4, 5]$ & $[1, 2, 3, 4, 5]$  & $[1, 2, 3, 4, 5]$\\
			$[5, 4, 3, 2, 1]$  & $[1, 2, 3, 4, 5]$ & $[1, 2, 3, 4, 5]$\\
			$[9, 7, 5, 1, 4]$  & $[1, 4, 5, 7, 9]$  & $[1, 4, 5, 7, 9]$\\
			$[5]$  & $[5]$  & $[5]$\\
			$[]$  & $[]$  & $[]$\\
			\hline
		\end{tabular}
    \end{threeparttable}
	\end{center}
\end{table}


\section*{Вывод}

Были разработаны схемы всех трех алгоритмов сортировки. Для каждого алгоритма была вычислена трудоёмкость и оценены лучший и худший случаи трудоёмкости.
