\chapter{Конструкторская часть}
В этом разделе будут рассмотрены описания используемых типов данных, а также схемы алгоритмов конвейерной и линейной обработки матриц.

\section{Описание используемых типов данных}

При реализации алгоритмов будут использованы следующие типы данных:

\begin{itemize}
	\item количество задач (матриц) --- целое число;
	\item размер матрицы --- целое число;
	\item структура \textit{matrix\_s} --- содержит информацию о матрице --- элементы матрицы, её размер, а также информацию о найденном среднем арифметическом и максимальном элементах;
	\item структура \textit{queues\_t} --- содержит информацию об очередях.
\end{itemize}

\section{Разработка алгоритмов}
На рисунке \ref{img:linear} представлена схема алгоритма линейной обработки матрицы, на рисунке \ref{img:main_thread} --- схема алгоритма конвейерной обработки матрицы, а на рисунках \ref{img:thread1}--\ref{img:thread3} --- схемы потоков обработки матрицы (ленты конвейера).
\clearpage
\img{220mm}{linear}{Схема алгоритма линейной обработки матрицы}
\clearpage
\img{220mm}{main_thread}{Схема конвейерной обработки матрицы}
\clearpage
\img{220mm}{thread1}{Схема 1 потока обработки матрицы --- нахождение среднего арифметического}
\clearpage
\img{220mm}{thread2}{Схема 2 потока обработки матрицы --- нахождение максимального элемента}
\clearpage
\img{220mm}{thread3}{Схема 3 потока обработки матрицы --- заполнения матрицы новыми значенями}
\clearpage


\section{Классы эквивалентности при функциональном тестировании}

Для функционального тестирования выделены классы эквивалентности, представленные ниже.

\begin{enumerate}
	\item Неверно выбран пункт меню --- не число или число, меньшее 0 или большее 4.
	\item Неверно введено количество матриц --- не число или число, меньшее 1.
	\item Неверно введен размер матриц --- не число или число, меньшее 2.
	\item Корректный ввод всех параметров.
\end{enumerate}


\section*{Вывод}

В данном разделе были построены схемы алгоритмов, рассматриваемых в лабораторной работе, были описаны классы эквивалентности для тестирования, структура программы.
