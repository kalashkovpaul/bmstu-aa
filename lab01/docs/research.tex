\chapter{Исследовательская часть}

В данном разделе будут приведены примеры работы программы, а также проведен сравнительный анализ алгоритмов при различных ситуациях на основе полученных данных.

\section{Технические характеристики}

Технические характеристики устройства, на котором выполнялось тестирование представлены далее:

\begin{itemize}[label=---]
    \item операционная система: Mac OS Monterey Версия 12.5.1 (21G83) \cite{macos} x86\_64;
    \item память: 16 Гб;
    \item процессор: 2,7 ГГцz 4‑ядерный процессор Intel Core i7 \cite{intel}.
\end{itemize}

При тестировании ноутбук был включен в сеть электропитания. Во время тестирования ноутбук был нагружен только встроенными приложениями окружения, а также системой тестирования.

\section{Демонстрация работы программы}

На рисунке \ref{img:example} представлен пример работы программы: возможность выбора используемого алгоритма, ввода двух строк, просмотра матрицы вычисления расстояния и результата работы реализованного алгоритма.

\imgScale{0.75}{example}{Пример работы программы}
\clearpage

\section{Время выполнения алгоритмов}

Для замера времени используется функция замера процессорного времени process\_time(...) из библиотеки time на Python. Она возвращает пользовательское процессорное время типа float.

Использовать функцию приходится дважды, затем из конечного времени нужно вычесть начальное, чтобы получить результат.

Замеры проводились для длины слова от 0 до 9 по 100 раз на различных входных данных.

Результаты замеров приведены в таблице \ref{tbl:time_mes} (время в мс).

\begin{table}[h]
    \begin{center}
        \begin{threeparttable}
        \captionsetup{justification=raggedright,singlelinecheck=off}
        \caption{Результаты замеров времени}
        \label{tbl:time_mes}
        \begin{tabular}{|c|c|c|c|c|}
            \hline
            Длина & Л.(матр) & Д-Л.(матр.)& Д-Л.(рек.) & Д.-Л.(рек. с кешем)  \\
            \hline
            0 & 0.0033 & 0.0074 & 0.0089 & 0.0032 \\
            \hline
            1 & 0.0138 & 0.0130 & 0.0091 & 0.0216 \\ 
            \hline
            2 & 0.0154 & 0.0169 & 0.0326 & 0.0363 \\ 
            \hline
            3 & 0.0225 & 0.0227 & 0.1430 & 0.0521 \\ 
            \hline
            4 & 0.0284 & 0.0331 & 0.6516 & 0.0751 \\ 
            \hline
            5 & 0.0410 & 0.0472 & 3.1557 & 0.1328 \\ 
            \hline
            6 & 0.0509 & 0.0633 & 16.7735 & 0.1755 \\ 
            \hline
            7 & 0.0653 & 0.0854 & 89.8081 & 0.2375 \\ 
            \hline
            8 & 0.0866 & 0.1064 & 496.2408 & 0.3080 \\ 
            \hline
            9 & 0.1049 & 0.1354 & 2724.1102 & 0.3792 \\ 
            \hline
		\end{tabular}
    \end{threeparttable}
\end{center}
\end{table}

Также на рисунках \ref{img:graph_lev}, \ref{img:graph_dam_lev}, \ref{img:graph_dam_lev_two} приведены графические результаты замеров.

\imgHeight{100mm}{graph_lev}{Результат работы алгоритма нахождения расстояния Левештейна (матричного)}

\imgHeight{100mm}{graph_dam_lev}{Сравнение алгоритмов нахождения расстояния Дамерау-Левенштейна (матричного, рекурсивного и рекурсивного c использованием кеша)}

\imgHeight{100mm}{graph_dam_lev_two}{Сравнение алгоритмов нахождения расстояния Дамерау-Левенштейна (матричного и рекурсивного c использованием кеша)}
\clearpage

Сложность матричного алгоритма нахождения расстояния Левенштейна составляет $O(n^2)$ (рисунок \ref{img:graph_lev}).

В общем случае рекурсивный алгоритм алгоритм нахождения расстояния Дамерау-Левенштейна медленнее, чем его реализация с кешем или матричная реализация (рисунок \ref{img:graph_dam_lev}), а также что матричная реализация нескольлко быстрее рекурсивного алгоритма с использованием кеша (рисунок \ref{img:graph_dam_lev_two}).


\section{Вывод}

Исходя из замеров по памяти, итеративные алгоритмы проигрывают рекурсивным, потому что максимальный размер памяти в них растет, как произведение длин строк, а в рекурсивных -- как сумма длин строк.

В результате проведенных измерений было получено, что обычно матричный алгоритм нахождения расстояния Дамерау-Левенштейна быстрее рекурсивного алгоритма с использованием кеша, однако занимает он намного больше памяти. Так, для длины слова в 9 символов матричная реализация быстрее рекурсивной в 2 раза, однако занимает в 5 раз больше памяти

Также было выявлено, что на длине строк в 4 символа рекурсивная реализация алгоритма Дамерау-Левенштейна в уже в 21 раз медленнее матричной реализации алгоритма. При увеличении длины строк в геометрической прогрессии растет и время работы рекурсивной реализации. Следовательно, для строк длиной более 4 символов стоит использовать матричную реализацию.
