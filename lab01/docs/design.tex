\chapter{Конструкторская часть}
В этом разделе представлены описания используемых типов данных, а также схемы алгоритмов вычисления расстояния Левенштейна и Дамерау-Левенштейна.

\section{Описание используемых типов данных}
При реализации алгоритмов будут использованы следующие структуры данных:

\begin{itemize}
    \item две переменных строкового типа;
    \item длина строки -- целое число;
    \item в матричной реализации алгоритмов поиска расстояний Левенштейна и Дамерау-Левенштейна, а также рекурсивной реализации с кешем -- матрица, которая является двумерным списком целочисленного типа.
\end{itemize}


\section{Сведения о модулях программы}
Программа состоит из шести модулей:
\begin{itemize}
	\item $main.py$ -- файл, содержащий точку входа;
    \item $menu.py$ -- файл, содержащий код меню программы;
    \item $test.py$ -- файл, содержаший код тестирования алгоритмов;
    \item $utils.py$ -- файл, содержащий служебные алгоритмы;
    \item $constants.py$ -- файл, содержаший константы программы;
    \item $algorythms.py$ -- файл, содержащий код всех алгоритмов. \newline
\end{itemize}


\section{Разработка алгоритмов}
На рисунках \ref{img:lev}-\ref{img:dam_lev} представлены схемы алгоритмов вычисления расстояния Левенштейна и Дамерау-Левенштейна.

\imgScale{0.9}{lev}{Схема матричного алгоритма нахождения расстояния Левенштейна}
\imgScale{1}{dam_lev}{Схема матричного алгоритма нахождения расстояния Дамерау-Левенштейна}
\imgScale{0.75}{dam_lev_rec}{Схема рекурсивного алгоритма нахождения расстояния Дамерау-Левенштейна}
\imgScale{0.75}{dam_lev_rec_cache}{Схема рекурсивного алгоритма нахождения расстояния Дамерау-Левенштейна с использованием кеша (матрицы)}

\clearpage

\section{Классы эквивалентности функционального тестирования}

Для функционального тестирования выделены классы эквивалентности, представленные ниже.

\begin{enumerate}
    \item Ввод двух пустых строк.
    \item Одна из строк -- пустая.
    \item Расстояния, которые вычислены алгоритмами Левенштейна и Дамерау-Левенштейна, равны.
    \item Расстояния, которые вычислены алгоритмами Левенштейна и Дамерау-Левенштейна, отличаются
\end{enumerate}

\section{Использование памяти}

Замеры времени работы и используемой памяти алгоритмов Левенштейна и Дамерау-Левенштейна могут быть произведены одним и тем же способом.
Тогда рассмотрим только рекурсивную и матричную реализации данных алгоритмов.

Пусть n -- длина строки S1, m -- длина строки S2.

Тогда затраты по памяти будут такими:
\begin{itemize}
    \item алгоритм нахождения расстояния Левенштейна (матричный):

    \begin{itemize}
        \item для матрицы -- ((n + 1)$ \cdot$ (m + 1)) $ \cdot$ sizeof(int));
        \item для S1, S2 -- (n + m) $ \cdot$ sizeof(char);
        \item для n, m -- 2 $ \cdot$ sizeof(int);
        \item доп. переменные -- 3 $ \cdot$ sizeof(int);
        \item адрес возврата.
    \end{itemize}

    \item алгоритм нахождения расстояния Дамерау-Левенштейна (матричный):

    \begin{itemize}
        \item для матрицы -- ((n + 1) $ \cdot$ (m + 1)) $ \cdot$ sizeof(int));
        \item для S1, S2 -- (n + m) $ \cdot$ sizeof(char);
        \item для n, m -- 2 $ \cdot$ sizeof(int);
        \item доп. переменные -- 4 $ \cdot$ sizeof(int);
        \item адрес возврата.
    \end{itemize}

    \item алгоритм нахождения расстояния Дамерау-Левенштейна (рекурсивный), где для каждого вызова:

    \begin{itemize}
        \item для S1, S2 -- (n + m) $ \cdot$ sizeof(char);
        \item для n, m -- 2 $ \cdot$ sizeof(int);
        \item доп. переменные - 2 $ \cdot$ sizeof(int);
        \item адрес возврата.
    \end{itemize}

    \item алгоритм нахождения расстояния Дамерау-Левенштейна с использованием кеша в виде матрицы (память на саму матрицу: ((n + 1) $ \cdot$ (m + 1)) $ \cdot$ sizeof(int)) (рекурсивный), где для каждого вызова:

    \begin{itemize}
        \item для S1, S2 -- (n + m) $ \cdot$ sizeof(char);
        \item для n, m -- 2 $ \cdot$ sizeof(int);
        \item доп. переменные -- 2 $ \cdot$ sizeof(int);
        \item ссылка на матрицу - 8 байт;
        \item адрес возврата.
    \end{itemize}

\end{itemize}

\section{Вывод}
В данном разделе были представлены описания используемых типов данных, а также схемы алгоритмов, рассматриваемых в лабораторной работе. 
Можно сделать вывод, что алгоритмы нахождения расстояния Левенштейна, использующие матрицу (матричный подход), а также рекурсивные алгоритмы с кешем, используют значительно больше памяти, чем рекурсивная реализация (примерно на  (n + m) $ \cdot$ sizeof(char) байт - размер используемой матрицы/кеша).