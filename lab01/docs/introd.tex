\chapter*{Введение}
\addcontentsline{toc}{chapter}{Введение}

Операции работы со строками являются важной частью всего программирования. Часто возникает потребность в использовании строк для различных задач, для решения которых нужны алгоритмы сравнения строк, о которых и пойдет речь в данной работе.
Они используются при решении таких задач, как исправление ошибок в тексте, предлагая заменить введенное слово с ошибкой на наиболее подходящее.



\textbf{Целью данной работы} является изучение, реализация и исследование алгоритмов нахождения расстояний Левенштейна и Дамерау-Левенштейна.
Для достижения поставленной цели необходимо выполнить следующие задачи:
\begin{itemize}
	\item изучить и реализовать алгоритмы нахождения расстояния Левенштейна и Дамерау-Левенштейна;
    \item протестроивать по времени и по памяти алгоритмы нахождения расстояния Левенштейна и Дамерау-Левенштейна;
    \item сравнить по времени работы алгоритмов нахождения расстояния Левенштейна и Дамерау-Левенштейна;
    \item проанализировать по времени работы матричной, рекурсивной, а также рекурсивной с использованием кеша реализаций алгоритма нахождения расстояния Дамерау-Левенштейна;
	\item описать и обосновать полученные результаты в отчете о выполненной лабораторной работе, выполненном как расчётно-пояснительная записка к работе.
\end{itemize}
