\chapter*{Заключение}
\addcontentsline{toc}{chapter}{Заключение}

В результате исследования было определено, что время алгоритмов нахождения расстояний Левенштейна и Дамерау-Левенштейна растет в геометрической прогрессии при увеличении длин строк. Лучшие показатели по времени дает матричная реализация алгоритма нахождения расстояния Дамерау-Левенштейна и его рекурсивная реализация с кешем, использование которых приводит к 21-кратному превосходству по времени работы уже на длине строки в 4 символа за счет сохранения необходимых промежуточных вычислений. При этом матричные реализации занимают довольно много памяти при большой длине строк.


Цель, которая была поставлена в начале лабораторной работы была достигнута, а также в ходе выполнения лабораторной работы были решены следующие задачи:

\begin{itemize}
	\item были изучены и реализованы алгоритмы нахождения расстояния Левенштейна и Дамерау-Левенштейна;
	\item были также изучены матричная реализация, а также реализация с использованием кеша в виде матрицы для алгоритма нахождения расстояния Дамерау-Левенштейна;
    \item проведен сравнительный анализ алгоритмов нахождения расстояний Дамерау-Левенштейна в матричной, рекурсивной и рекурсивной и использованием кэша реализациях;
	\item подготовлен отчет о лабораторной работе.
\end{itemize}