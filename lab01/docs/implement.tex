\chapter{Технологическая часть}

В данном разделе будут рассмотрены средства реализации, а также представлены листинги алгоритмов определения расстояния Левенштейна и Дамерау-Левенштейна.

\section{Средства реализации}
В данной работе для реализации был выбран язык программирования $Python \cite{python-lang}$. В текущей лабораторной работе требуется замерить процессорное время для выполняемой программы, а также построить графики. Все эти инструменты присутствуют в выбранном языке программирования.

Время работы было замерено с помощью функции \textit{process\_time(...)} из библиотеки $time \cite{python-lang-time}$.


\section{Листинги кода}

В листингах \ref{lst:lev_mat}-\ref{lst:dam_lev_rec_cache} представлена реализация алгоритмов нахождения расстояния Левенштейна и Дамерау-Левенштейна.

\clearpage

\begin{center}
    \captionsetup{justification=raggedright,singlelinecheck=off}
    \begin{lstlisting}[label=lst:lev_mat,caption=Алгоритм нахождения расстояния Левенштейна (матричный)]
def levenshteinDistance(str1: str, str2: str, output: bool = True) -> int:
	n = len(str1)
	m = len(str2)
	matrix = createLevenshteinMatrix(n + 1, m + 1)
	
	for i in range(1, n + 1):
		for j in range(1, m + 1):
		add = matrix[i - 1][j] + 1
		delete = matrix[i][j - 1] + 1
		change = matrix[i - 1][j - 1]
		
		if (str1[i - 1] != str2[j - 1]):
			change += 1
		
		matrix[i][j] = min(add, delete, change)
	
	if (output):
		printMatrix(matrix, str1, str2)
	
	return matrix[n][m]
\end{lstlisting}
\end{center}


\begin{center}
    \captionsetup{justification=raggedright,singlelinecheck=off}
    \begin{lstlisting}[label=lst:dam_lev_mat,caption=Алгоритм нахождения расстояния Дамерау-Левенштейна (матричный)]
def damerauLevenshteinDistance(str1: str, str2: str, output: bool = True) -> int:
	n = len(str1)
	m = len(str2)
	matrix = createLevenshteinMatrix(n + 1, m + 1)
	
	for i in range(1, n + 1):
		for j in range(1, m + 1):
			add = matrix[i - 1][j] + 1
			delete = matrix[i][j - 1] + 1
			change = matrix[i - 1][j - 1]
			if (str1[i - 1] != str2[j - 1]):
				change += 1
			swap = n
			if (i > 1 and j > 1 and
				str1[i - 1] == str2[i - 2] and
				str1[i - 2] == str2[i - 1]):
					swap = matrix[i - 2][j - 2] + 1
		
		matrix[i][j] = min(add, delete, change, swap)
	
	if (output):
		printMatrix(matrix, str1, str2)
	
	return matrix[n][m]
\end{lstlisting}
\end{center}


\begin{center}
    \captionsetup{justification=raggedright,singlelinecheck=off}
    \begin{lstlisting}[label=lst:dam_lev_rec,caption=Алгоритм нахождения расстояния Дамерау-Левенштейна (рекурсивный)]
def damerauLevenshteinDistanceRecursive(str1: str, str2: str, output: bool = True) -> int:
	n = len(str1)
	m = len(str2)
	flag = 0
	if ((n == 0) or (m == 0)):
	return abs(n - m)
	
	if (str1[-1] != str2[-1]):
	flag = 1
	
	minDistance = min(
		damerauLevenshteinDistanceRecursive(str1[:-1], str2) + 1,
		damerauLevenshteinDistanceRecursive(str1, str2[:-1]) + 1,
		damerauLevenshteinDistanceRecursive(str1[:-1], str2[:-1]) + flag
	)
	if (n > 1 and m > 1 and str1[-1] == str2[-2] and str1[-2] == str2[-1]):
		minDistance = min(
			minDistance,
			damerauLevenshteinDistanceRecursive(str1[:-2], str2[:-2]) + 1
	)
	
	return minDistance
\end{lstlisting}
\end{center}


\begin{center}
    \captionsetup{justification=raggedright,singlelinecheck=off}
    \begin{lstlisting}[label=lst:dam_lev_rec_cache,caption=Алгоритм нахождения расстояния Дамерау-Левенштейна (рекурсивный с использованием кеша)]
def recursiveWithCache(
str1: str, str2: str, n: int, m: int, matrix: Matrix) -> None:
	if (matrix[n][m] != -1):
		return matrix[n][m]
	if (n == 0):
		matrix[n][m] = m
		return matrix[n][m]
	if ((n > 0) and (m == 0)):
		matrix[n][m] = n
		return matrix[n][m]
	
	add = recursiveWithCache(str1, str2, n - 1, m, matrix) + 1
	delete = recursiveWithCache(str1, str2, n, m - 1, matrix) + 1
	change = recursiveWithCache(str1, str2, n - 1, m - 1, matrix)
	if (str1[n - 1] != str2[m - 1]):
		change += 1 # flag
	
	matrix[n][m] = min(add, delete, change)
	if (n > 1 and m > 1 and
		str1[n - 1] == str2[m - 2] and
		str1[n - 2] == str2[m - 1]):
	
			matrix[n][m] = min(
			matrix[n][m],
			recursiveWithCache(str1, str2, n - 2, m - 2, matrix) + 1
			)
	
	return matrix[n][m]

def damerauLevenshteinDistanceRecurciveCache(
str1: str, str2: str, output: bool = True) -> int:
	n = len(str1)
	m = len(str2)
	matrix = createLevenshteinMatrix(n + 1, m + 1)
	
	for i in range(n + 1):
		for j in range(m + 1):
			matrix[i][j] = -1
		
	recursiveWithCache(str1, str2, n, m, matrix)
	
	if (output):
		printMatrix(matrix, str1, str2)
	
	return matrix[n][m]

\end{lstlisting}
\end{center}

\section{Функциональные тесты}

В таблице \ref{tbl:functional_test} приведены тесты для функций, реализующих алгоритмы нахождения расстояния Левенштейна и Дамерау-Левенштейна. Тесты \textit{для всех алгоритмов} пройдены успешно.

\begin{table}[h]
	\begin{center}
        \begin{threeparttable}
        \captionsetup{justification=raggedright,singlelinecheck=off}
		\caption{\label{tbl:functional_test} Функциональные тесты}
		\begin{tabular}{|c|c|c|c|c|}
			\hline
			& & & \multicolumn{2}{c|}{Ожидаемый результат} \\
			\hline
			№&Строка 1&Строка 2&Левенштейн&Дамерау-Л. \\
			\hline
            1&"пустая строка"&"пустая строка"&0&0 \\
            \hline
            2&"пустая строка"&слово&5&5 \\
            \hline
            3&проверка&"пустая строка"&8&8 \\
            \hline
            4&ремонт&емонт&1&1 \\
			\hline
			5&гигиена&иена&3&3 \\
			\hline
            6&нисан&автоваз&6&6 \\
			\hline
			7&спасибо&пожалуйста&9&9 \\
			\hline
            8&что&кто&1&1 \\
			\hline
            9&ты&тыква&3&3 \\
			\hline
            10&есть&кушать&4&4 \\
			\hline
			11&abba&baab&3&2 \\
			\hline
            12&abcba&bacab&4&2 \\
			\hline
		\end{tabular}
        \end{threeparttable}
	\end{center}
\end{table}

\section{Вывод}

Были представлены всех алгоритмов нахождения расстояния Левенштейна и Дамерау-Левенштейна, которые были описаны в предыдущем разделе. Также в данном разделе была приведена информации о выбранных средствах для разработки алгоритмов.
