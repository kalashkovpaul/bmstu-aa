\chapter{Аналитическая часть}
В этом разделе будет представлена информация о формализации объекта <<человек>> и его признака <<рост>> по варианту, составлена анкета для заполнения респондентами, проведено анкетирование респондентов и построена функция принадлежности термам числовых значений признака, описываемого лингвистической переменной, на основе статистической обработки мнений респондентов, выступающих в роли экспертов.
Также будет описаны структура данных словаря и алгоритм поиска в словаре. 

\section{Формализация объекта и его признака}
\label{formal}
Согласно согласованному варианту, формализуем объект <<человек>> следующим образом: определим набор данных и признак объекта, на основании которого составим набор термов.
Набор данных:
\begin{enumerate}[label=\arabic*)]
	\item имя человека --- строка;
	\item пол человека --- строка;
	\item родная страна человека --- строка;
	\item профессия человека --- строка.
\end{enumerate}
Согласно варианту, признаком, по которому будет производиться поиск объектов, будет являться \textit{рост} в сантиметрах --- целое число.

Определим следующие термы, соответствующие признаку <<рост>>:
\begin{enumerate}[label=\arabic*)]
	\item <<Очень низкий>>;
	\item <<Низкий>>;
	\item <<Средний>>;
	\item <<Высокий>>;
	\item <<Очень высокий>>.
\end{enumerate}

Также введём универсальное для данной задачи множество оцениваемой величины (роста) $H$:
\begin{equation}
	\label{eq:h}
	H = \{80, 90, 100, 110, 120, 130, 140, 150, 160, 170, 180, 190, 200, 210, 220\}
\end{equation}

\section{Анкетирование респондентов}

Было проведено анкетирование следующих респондентов:
\begin{enumerate}[label=\arabic*)]
	\item Соловов Юрий, группа ИУ7-56Б --- Респондент 1;
	\item Чепрасов Кирилл, группа ИУ7-56Б --- Респондент 2;
	\item Виноградов Алексей, группа ИУ7-56Б --- Респондент 3;
	\item Авсюнин Алексей, группа ИУ7-56Б --- Респондент 4;
	\item Ковель Александр, группа ИУ7-56Б --- Респондент 5;
	\item Комаров Никита, группа ИУ7-52Б --- Респондент 6.
\end{enumerate}

Респонденты, выступающие в качестве экспертов, для каждого из приведённых выше термов указали соответствующий промежуток, элементами которого являются числа из введённого для поставленной задачи множества оценимоемой величины.

Результаты анкетирования перечисленных респондентов продемонстрированы в таблице~\ref{tbl:anket}. В данной таблице Респ. --- сокращение от <<Респондент>>, термы 1 -- 5 --- термы, соответствующие обозначенным в п. \ref{formal} термам.

\begin{center}
	\captionsetup{justification=raggedright,singlelinecheck=off}
	\begin{longtable}[c]{|c|c|c|c|c|c|c|}
		\caption{Результаты анкетирования\label{tbl:anket}}\\ \hline
		Терм & Респ. 1 & Респ. 2 & Респ. 3  & Респ. 4 & Респ. 5 & Респ.6 \\ \hline
		1 &   [80; 110) &   [80; 140] & [80; 150]  & [80; 140] & [80; 130] & [80; 149] \\ \hline
		2 &   [110; 150) &   (140; 160] & (150; 165) & (140; 160] &  (130; 150] & [150; 164]\\ \hline
		3&   [150; 180) &   (160; 180] & [165; 190] & (160 ; 180]  & (150; 170] & [165; 179]\\ \hline
		4 &   [180; 190) &  (180; 190] & [190; 200] & (180; 195] & (170; 190] & [180; 195]\\ \hline
		5 &   [190; 220]  &  (190; 220] & (200; 220] & (195; 220] &  (190; 220] & [196; 220]\\ \hline
	\end{longtable}
\end{center}

\subsection{Построение функции принадлежности термам}

Построим графики функций принадлежности числовых значений переменной термам, описывающим группы значений лингвистической переменной.

Для этого для каждого значения из $H$ для каждого терма из перечисленных найдём количество респондентов, согласно которым значение из $H$ удоволетворяет сопоставляемому терму.
Данное значение поделим на количество респондентов --- это и будет значением функции $\mu$ для терма в точке.
Графики функций принадлежности числовых значений роста термам, приведён на рисунке \ref{img:function}.
	

\imgHeight{100mm}{function}{Графики функций принадлежности числовых значений переменной термам, описывающим группы значений лингвистической переменной}

В соответствии с полученным графиком будем считать людей:
\begin{enumerate}[label=\arabic*)]
	\item очень низкого роста, если их значение их роста лежит в промежутке $[80; 143]$ сантиметров;
	\item низкого роста, если их значение их роста лежит в промежутке $[144; 162]$ сантиметров;
	\item среднего роста, если их значение их роста лежит в промежутке $[163; 182]$ сантиметров;
	\item высокого роста, если их значение их роста лежит в промежутке $[183; 194]$ сантиметров;
	\item очень высокого роста, если их значение их роста лежит в промежутке $[195; 220]$ сантиметров.
\end{enumerate}

\section{Словарь}

\textbf{Словарь} (англ. \textit{dictionary})~\cite{dict} --- тип данных, который позволяет хранить пары вида <<ключ  --- значение>> (англ. \textit{key --- value}) $(k, v)$. Он поддерживает три операции:
\begin{enumerate}[label=\arabic*)]
	\item добавление пары;
	\item поиск по ключу;
	\item удаление по ключу.
\end{enumerate}
В паре $(k, v)$ $v$ --- значение, ассоциируемое с $k$.

\subsection{Алгоритм поиска в словаре}

Рассмотрим алгоритм поиска в словаре объектов, удоволетворяющих ограничению, заданному в вопросе на ограниченном естественном языке.
Предлагается использовать метод полного перебора, рассмотренный ниже.

\textbf{Метод полного перебора}~\cite{full-comb} --- метод решения поиска значения в словаре, при котором поочерёдно перебираются все ключи словаря, пока не будет найден нужный.

Чем дальше искомый ключ от начала словаря, тем выше трудоемкость алгоритм. Так, если на старте алгоритм затрагивает $b$ операций, а при сравнении $k$ операций, то:\newline
\begin{itemize}[label=---]
	\item элемент найден на первом сравнении за $b + k$ операций (лучший случай);
	\item элемент найден на $i$-ом сравнении за $b + i \cdot k$ операций;
	\item элемент найден на последнем сравнении за $b +  N \cdot k$ операций, где $N$ -- размер словаря (худший случай).
\end{itemize}

Если обозначить трудоёмкость выполняемых операций равной единице, то средняя трудоёмкость равна:

\begin{equation}
	f = b + k \cdot \left(1 + \frac{N}{2} - \frac{1}{N + 1}\right)
\end{equation}

Прежде, чем искать необходимые значения для отдельных ключей, необходимо выделить набор отдельных ключей как ограничение, заданное на ограниченном естественном языке (русском). 
Рассмотрим следующие этапы выделения ограничения:
\begin{enumerate}[label=\arabic*)]
	\item найти вхождение слова, описывающего объект (т.~е. в запросе должна идти речь о людях);
	\item найти вхождение слов или словосочетаний, указывающих на признак (т.~е. в запросе должна идти речь о росте);
	\item найти терм.
 \end{enumerate}

Для каждого из пунктов определим набор слов и словосочетаний, которые должны присутствовать в запросе для того, чтобы он считался корректным,~т.~е. гарантированно содержал информацию, достаточную для определения терма, поиск по которому будет осуществляться в словаре.

Для определения, что в запросе идёт речь об объекте <<человек>> достаточно, чтобы в нём присутствовал ходя бы один из приведённых далее элементов: <<человек>>, <<людей>>, <<люди>>.

Для определения, что в запросе идёт речь о росте, в запросе должно присутствовать слово с корнем<<рост>>.

Для определения терма, о котором идёт речь в запросе, необходимо проверить присутствие следующих элементов для каждого из термов соответственно:\newline
\begin{enumerate}[label=\arabic*)]
	\item <<очень низкого>>;
	\item <<низкого>> (причём перед этим словом не стоит слово <<очень>>);
	\item <<среднего>>;
	\item <<высокого>> (причём перед этим словом не стоит слово <<очень>>);
	\item <<очень высокого>>;
\end{enumerate}

Также необходимо обработать ситуации, в которых перед выделенными словами или словосочетаниями стоит слово <<не>> --- это будет говорить о том, что необходимо найти все объекты, не соответствующие указанному терму.

Запросы, в которых отсутствует информация о термах, идёт речь не об объектах типа <<человек>> или не об их росте, или в которых присутствует информация о нескольких термах сразу, будут считаться некорректными запросами.


\section{Требования к программе}
Требуется разработать программу, которая по словарю, в котором ключами являются численные характеристики роста человека, а значением --- данные о человеке, а также пользовательскому запросу в виде строки, содержащей вопрос на ограниченном естественном языке (русском языке) с ограничением на признак роста, выдаст релевантные запросу объекты из словаря либо сообщение, что запрос не распознан либо не соответствует заданной тематике.

\section*{Вывод}

В этом разделе была представлена информация о формализации объекта <<человек>> и его признака <<рост>> по варианту, составлена анкета для заполнения респондентами, проведено анкетирование респондентов и построена функция принадлежности термам числовых значений признака, описываемого лингвистической переменной, на основе статистической обработки мнений респондентов, выступающих в роли экспертов.
Также были описаны структура данных словаря и алгоритм поиска в словаре. 
