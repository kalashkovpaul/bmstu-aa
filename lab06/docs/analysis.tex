\chapter{Аналитическая часть}
В этом разделе будет представлена информация о формализации объекта <<человек>> и его признака <<рост>> по варианту, составлена анкета для заполнения респондентами, проведено анкетирование респондентов и построена функция принадлежности термам числовых значений признака, описываемого лингвистической переменной, на основе статистической обработки мнений респондентов, выступающих в роли экспертов.
Также будет описаны структура данных словаря и алгоритм поиска в словаре. 

\section{Формализация объекта и его признака}
Согласно согласованному варианту, формализуем объект <<человек>> следующим образом: определим набор данных и признак объекта, на основании которого составим набор термов.
Набор данных:
\begin{enumerate}[label=\arabic*)]
	\item имя человека --- строка;
	\item пол человека --- строка;
	\item родная страна человека --- строка;
	\item профессия человека --- строка.
\end{enumerate}
Согласно варианту, признаком, по которому будет производиться поиск объектов, будет являться \textit{рост} в сантиметрах --- целое число.

Определим следующие термы, соответствующие признаку <<рост>>:
\begin{enumerate}[label=\arabic*)]
	\item <<Очень низкий>>;
	\item <<Низкий>>;
	\item <<Средний>>;
	\item <<Высокий>>;
	\item <<Очень высокий>>.
\end{enumerate}

Также введём универсальное для данной задачи множество оцениваемой величины (роста) $H$:
\begin{equation}
	\label{eq:h}
	H = \{80, 90, 100, 110, 120, 130, 140, 150, 160, 170, 180, 190, 200, 210, 220\}
\end{equation}

\section{Анкетирование респондентов}

Было проведено анкетирование следующих респондентов:
\begin{enumerate}[label=\arabic*)]
	\item Соловов Юрий, группа ИУ7-56Б --- Респондент 1;
	\item Чепрасов Кирилл, группа ИУ7-56Б --- Респондент 2;
	\item Виноградов Алексей, группа ИУ7-56Б --- Респондент 3;
	\item Авсюнин Алексей, группа ИУ7-56Б --- Респондент 4;
	\item Ковель Александр, группа ИУ7-56Б --- Респондент 5;
	\item Комаров Никита, группа ИУ7-52Б --- Респондент 6.
\end{enumerate}


\section{Метод на основе муравьиного алгоритма}

\textbf{Муравьиный алгоритм} (англ. \textit{ant colony optimization})  \cite{full-comb} --- метод решения задачи оптимизации, основаный на принципе поведения колонии муравьев.

Муравьи действуют, руководствуясь органами чувств. 
Каждый муравей оставляет на своём пути феромоны, чтобы другие могли ориентироваться. 
При большом количестве муравьев наибольшее количество феромона остаётся на наиболее посещаемом пути, посещаемость же может быть связана с длинами рёбер.

Суть в том, что отдельно взятый муравей мало что может, поскольку он способен выполнять только максимально простые задачи. Но при большом числе других таких муравьев они могут выступать самостоятельными вычислительными единицами. Муравьи используют непрямой обмен информацией через огружающую среду посредством феромона.

Пусть муравей имеет следующие характеристики:
\begin{enumerate}[label=\arabic*)]
	\item зрение --- способность определить длину ребра;
	\item память --- способность запомнить пройденный маршрут;
	\item обоняние --- способность чуять феромон.
\end{enumerate}


Также введем целевую функцию \eqref{d_func}, характеризующую привлекательность ребра, определяемую благодаря зрению.

\begin{equation}
	\label{d_func}
	\eta_{ij} = 1 / D_{ij},
\end{equation}
где $D_{ij}$ — расстояние от текущего пункта $i$ до заданного пункта $j$.


Также понадобится формула вычисления вероятности перехода в заданную точку \eqref{posib}.

\begin{equation}
	\label{posib}
	p_{k,ij} = \begin{cases}
		\frac{\eta_{ij}^{\alpha}\cdot\tau_{ij}^{\beta}}{\sum_{q\notin J_k} \eta^\alpha_{iq}\cdot\tau^\beta_{iq}}, j \notin J_k \\
		0, j \in J_k
	\end{cases}
\end{equation}
где $a$ --- параметр влияния длины пути, $b$ --- параметр влияния феромона, $\tau_{ij}$ --- количество феромонов на ребре $ij$, $\eta_{ij}$ --- привлекательность ребра $ij$, $J_k$ --- список посещённых за текущий день городов.

После завершения движения всех муравьев (ночью, перед наступлением следующего дня), феромон обновляется по формуле \eqref{update_phero_1}.
\begin{equation}
	\label{update_phero_1}
	\tau_{ij}(t+1) = \tau_{ij}(t)\cdot(1-p) + \Delta \tau_{ij}(t).
\end{equation}
При этом
\begin{equation}
	\label{update_phero_2}
	\Delta \tau_{ij}(t) = \sum_{k=1}^N \Delta \tau^k_{ij}(t),
\end{equation}
где
\begin{equation}
	\label{update_phero_3}
	\Delta\tau^k_{ij}(t) = \begin{cases}
		Q/L_{k}, \textrm{ребро посещено муравьем $k$ в текущий день $t$,} \\
		0, \textrm{иначе}
	\end{cases}
\end{equation}

Поскольку вероятность перехода в заданную точку \ref{posib} не должна быть равна нулю, необходимо обеспечить неравенство $\tau_{ij} (t)$ нулю посредством введения дополительного минимально возможного значения феромона $\tau_{min}$ и в случае, если $\tau_{ij} (t+1)$ принимает значение, меньшее $\tau_{min}$, откатывать феромон до этой величины. 


Путь выбирается по следующей схеме.
\begin{enumerate}
	\item Каждый муравей имеет список запретов --- список уже посещенных городов (вершин графа).
	\item Муравьиное зрение отвечает за эвристическое желание посетить вершину.
	\item Муравьиное обоняние отвечает за ощущение феромона на определенном пути (ребре). При этом количество феромона на пути (ребре) в день $t$ обозначается как $\tau_{i, j} (t)$.
	\item После прохождения определенного ребра муравей откладывает на нем некотрое количество феромона, которое показывает оптимальность сделанного выбора, это количество вычисляется по формуле \eqref{update_phero_3}.
\end{enumerate}

\section{Требования к программе}
Выделим ряд требований к разрабатываемой программе:
\begin{itemize}[label=---]
	\item программа должна получать на вход матрицу смежности, для которой можно будет выбрать один из алгоритмов поиска оптимальных путей --- полным перебором или муравьиным алгоритмом;
	\item программа должна позволять пользователю определять коэффициенты и количество дней для муравьиного алгоритма;
	\item программа должна давать возможность получить минимальную сумму пути, а также сам путь, используя один из алгоритмов. Также должна присутствовать возможность провести тестирование по времени выполнения для разных размеров матриц.
\end{itemize}

\section*{Вывод}

В данном разделе была рассмотрена задача коммивояжёра, а также полный перебор для её решения и муравьиный алгоритм. Были представлены требования к разрабатываемому программному обеспечению.
