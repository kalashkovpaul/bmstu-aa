\chapter{Аналитическая часть}
В этом разделе будет представлена информация о задаче коммивояжёра, а также о способах её решения --- полном переборе и муравьином алгоритме.

\section{Задача коммивояжёра}
\textbf{Задача коммивояжёра} \cite{task} \text{(англ. \textit{traveling salesman problem})} --- (задача о бродячем торговце) одна из самых важных задач всей транспортной логистики, в которой рассматриваются вершины графа, а также матрица смежности (для расстояния между вершинами). 
Задача заключается в том, чтобы найти такой порядок посещения вершин графа, при котором путь будет минимален, каждая вершина будет посещена лишь один раз, а возврат произойдет в начальную вершину. 

\section{Алгоритм полного перебора для решения задачи коммивояжёра}

\textbf{Полный перебор для задачи коммивояжёра}\cite{full-comb} имеет высокую сложность алгоритма ($n!$). 
Суть в полном переборе всех возможных путей в графе и выбор наименьшего из них. 
Решение будет получено, но имеются большие затраты по времени выполнения при уже небольшом количестве вершин в графе.


\section{Муравьиный алгоритм}

\textbf{Муравьиный алгоритм} (англ. \textit{ant colony optimization})  \cite{full-comb} --- алгоритм, основаный на принципе поведения колонии муравьев.

Муравьи действуют, руководствуясь органами чувств. 
Каждый муравей оставляет на своём пути феромоны, чтобы другие могли ориентироваться. 
При большом количестве муравьев наибольшее количество феромона остаётся на оптимальном пути.

Суть в том, что отдельно взятый муравей мало что может, поскольку он способен выполнять только максимально простые задачи. Но при большого числа других таких муравьев они могу самоорганизовываться в большие очереди для решения сложных задач. Муравьи используют непрямой обмен информацией через огружающую среду посредством феромона.

Пусть муравей имеет следующие характеристики:
\begin{enumerate}[label=\arabic*)]
	\item зрение -- способен определить длину ребра;
	\item память -- запоминает пройденный маршрут;
	\item обоняние -- чувствует феромон.
\end{enumerate}


Также введем целевую функцию \eqref{d_func}, характеризующую привлекательность ребра, определяемую благодаря зрению.

\begin{equation}
	\label{d_func}
	\eta_{ij} = 1 / D_{ij},
\end{equation}
где $D_{ij}$ — расстояние из текущего пункта $i$ до заданного пункта $j$.


А также понадобится формула вычисления вероятности перехода в заданную точку \eqref{posib}.

\begin{equation}
	\label{posib}
	p_{k,ij} = \begin{cases}
		\frac{\eta_{ij}^{\alpha}\tau_{ij}^{\beta}}{\sum_{q\notin J_k} \eta^\alpha_{iq}\tau^\beta_{iq}}, j \notin J_k \\
		0, j \in J_k
	\end{cases}
\end{equation}
где $a$ --- параметр влияния длины пути, $b$ --- параметр влияния феромона, $\tau_{ij}$ --- расстояния от города $i$ до $j$, $\eta_{ij}$ --- количество феромонов на ребре $ij$, $J_k$ --- список посещённых за день городов.

После завершения движения всех муравьев, феромон обновляется по формуле \eqref{update_phero_1}:
\begin{equation}
	\label{update_phero_1}
	\tau_{ij}(t+1) = \tau_{ij}(t)(1-p) + \Delta \tau_{ij}.
\end{equation}
При этом
\begin{equation}
	\label{update_phero_2}
	\Delta \tau_{ij} = \sum_{k=1}^N \Delta \tau^k_{ij},
\end{equation}
где
\begin{equation}
	\label{update_phero_3}
	\Delta\tau^k_{ij} = \begin{cases}
		Q/L_{k}, \textrm{ребро посещено k-ым муравьем,} \\
		0, \textrm{иначе}
	\end{cases}
\end{equation}


Путь выбирается по следующей схеме:

\begin{enumerate}
	\item Каждый муравей имеет список запретов --- список уже посещенных городов (вершин графа).
	\item Муравьиной зрение отвечает за желание посетить вершину.
	\item Муравьиное обоняние отвечает за ощущение феромона на определенном пути (ребре). При этом количество феромона на пути (ребре) в момент времени $t$ обозначается как $\tau_{i, j} (t)$.
	\item После прохождения определенного ребра муравей откладывает на нем некотрое количество феромона, которое показывает оптимальность сделанного выбора (это кол-во вычисляется по формуле \eqref{update_phero_3}).
\end{enumerate}



\section*{Вывод}

В данном разделе была рассмотрена задача коммивояжёра, а также полный перебор для её решения и муравьиный алгоритм.
