\chapter{Исследовательская часть}

В данном разделе приведена постановка эксперимента.

\section{Формализация объекта и его признака}
\label{formal}
Согласно согласованному варианту, формализуем объект <<человек>> следующим образом: определим набор данных и признак объекта, на основании которого составим набор термов.
Набор данных:
\begin{enumerate}[label=\arabic*)]
	\item имя человека --- строка;
	\item пол человека --- строка;
	\item родная страна человека --- строка;
	\item профессия человека --- строка.
\end{enumerate}
Согласно варианту, признаком, по которому будет производиться поиск объектов, будет являться \textit{рост} в сантиметрах --- целое число.

Определим следующие термы, соответствующие признаку <<рост>>:
\begin{enumerate}[label=\arabic*)]
	\item <<Очень низкий>>;
	\item <<Низкий>>;
	\item <<Средний>>;
	\item <<Высокий>>;
	\item <<Очень высокий>>.
\end{enumerate}

Также введём универсальное для данной задачи множество оцениваемой величины (роста) $H$:
\begin{equation}
	\label{eq:h}
	H = \{80, 90, 100, 110, 120, 130, 140, 150, 160, 170, 180, 190, 200, 210, 220\}
\end{equation}

\section{Анкетирование респондентов}

Было проведено анкетирование следующих респондентов:
\begin{enumerate}[label=\arabic*)]
	\item Соловов Юрий, группа ИУ7-56Б --- Респондент 1;
	\item Чепрасов Кирилл, группа ИУ7-56Б --- Респондент 2;
	\item Виноградов Алексей, группа ИУ7-56Б --- Респондент 3;
	\item Авсюнин Алексей, группа ИУ7-56Б --- Респондент 4;
	\item Ковель Александр, группа ИУ7-56Б --- Респондент 5;
	\item Комаров Никита, группа ИУ7-52Б --- Респондент 6.
\end{enumerate}

Респонденты, выступающие в качестве экспертов, для каждого из приведённых выше термов указали соответствующий промежуток, элементами которого являются числа из введённого для поставленной задачи множества оценимоемой величины.

Результаты анкетирования перечисленных респондентов продемонстрированы в таблице~\ref{tbl:anket}. В данной таблице Респ. --- сокращение от <<Респондент>>, термы 1 -- 5 --- термы, соответствующие обозначенным в п. \ref{formal} термам.

\begin{center}
	\captionsetup{justification=raggedright,singlelinecheck=off}
	\begin{longtable}[c]{|c|c|c|c|c|c|c|}
		\caption{Результаты анкетирования\label{tbl:anket}}\\ \hline
		Терм & Респ. 1 & Респ. 2 & Респ. 3  & Респ. 4 & Респ. 5 & Респ.6 \\ \hline
		1 &   [80; 110) &   [80; 140] & [80; 150]  & [80; 140] & [80; 130] & [80; 149] \\ \hline
		2 &   [110; 150) &   (140; 160] & (150; 165) & (140; 160] &  (130; 150] & [150; 164]\\ \hline
		3&   [150; 180) &   (160; 180] & [165; 190] & (160 ; 180]  & (150; 170] & [165; 179]\\ \hline
		4 &   [180; 190) &  (180; 190] & [190; 200] & (180; 195] & (170; 190] & [180; 195]\\ \hline
		5 &   [190; 220]  &  (190; 220] & (200; 220] & (195; 220] &  (190; 220] & [196; 220]\\ \hline
	\end{longtable}
\end{center}

\subsection{Построение функции принадлежности термам}

Построим графики функций принадлежности числовых значений переменной термам, описывающим группы значений лингвистической переменной.

Для этого для каждого значения из $H$ для каждого терма из перечисленных найдём количество респондентов, согласно которым значение из $H$ удоволетворяет сопоставляемому терму.
Данное значение поделим на количество респондентов --- это и будет значением функции $\mu$ для терма в точке.
Графики функций принадлежности числовых значений роста термам, приведён на рисунке \ref{img:function}.


\imgHeight{100mm}{function}{Графики функций принадлежности числовых значений переменной термам, описывающим группы значений лингвистической переменной}

В соответствии с полученным графиком будем считать людей:
\begin{enumerate}[label=\arabic*)]
	\item очень низкого роста, если их значение их роста лежит в промежутке $[80; 143]$ сантиметров;
	\item низкого роста, если их значение их роста лежит в промежутке $[144; 162]$ сантиметров;
	\item среднего роста, если их значение их роста лежит в промежутке $[163; 182]$ сантиметров;
	\item высокого роста, если их значение их роста лежит в промежутке $[183; 194]$ сантиметров;
	\item очень высокого роста, если их значение их роста лежит в промежутке $[195; 220]$ сантиметров.
\end{enumerate}