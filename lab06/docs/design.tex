\chapter{Конструкторская часть}
В данном разделе будут представлены требования к разрабатываемому программному обеспечению, описание используемых типов данных, а также схемы алгоритма полного перебора и муравьиного алгоритма и классы эквивалентности для функционального тестирования.

\section{Требования к ПО}
Выделен ряд требований к программе:
\begin{itemize}
	\item программа должна получать на вход матрицу смежности, для которой можно будет выбрать один из алгоритмов поиска оптимальных путей --- полным перебором или муравьиным алгоритмом;
    \item программа должна позволять пользователю определять коэффициенты и количество дней для муравьиного алгоритма;
    \item программа должна давать возможность получить минимальную сумму пути, а также сам путь, используя один из алгоритмов. Также должна присутствовать возможность провести тестирование по времени выполнения для разных размеров матриц.
\end{itemize}

\section{Описание используемых типов данных}
При реализации алгоритмов будут использованы следующие типы дан-
ных:
\begin{itemize}
	\item размер матрицы смежности --- целое число;
	\item имя файла --- строка;
	\item коэффициенты $\alpha, \beta$, \textit{k\_evaporation} --- действительные числа;
	\item матрица смежности --- матрица целых чисел.
\end{itemize}

\section{Разработка алгоритмов}
На рисунке \ref{img:full_comb} представлена схема алгоритма полного перебора путей, а на рисунках \ref{img:ant_alg_part1}--\ref{img:ant_alg_part2} схема муравьиного алгоритма поиска путей. Также на рисунках \ref{img:find_pos}--\ref{img:update_phero} представлены схемы вспомогательных функций для муравьиного алгоритма.

\imgScale{0.5}{full_comb}{Схема алгоритма полного перебора путей}
\imgScale{0.5}{ant_alg_part1}{Схема муравьиного алгоритма (часть 1)}
\imgScale{0.5}{ant_alg_part2}{Схема муравьиного алгоритма (часть 2)}
\imgScale{0.5}{find_pos}{Схема алгоритма нахождения массива вероятностных переходов в непосещенные города}
\imgScale{0.5}{rand_choice}{Схема алгоритма нахождения следующего города на основании рандома}
\imgScale{0.5}{update_phero}{Схема алгоритма обновления матрицы феромонов}

\clearpage

\section{Классы эквивалентности при функциональном тестировании}

Для функционального тестирования выделены классы эквивалентности, представленные ниже.

\begin{enumerate}
	\item Неверно выбран пункт меню - не число или число, меньшее 0 или большее 8.
	\item Неверно введены коэффициенты $\alpha$, $\beta$, \textit{evaporation\_koef} - не число или число, меньшее 0.
	\item Неверно введено кол-во дней - не число или число, меньшее 0.
	\item Неверно введен размер матрицы - не число или число, меньшее 2.
	\item Корректный ввод всех параметров.
\end{enumerate}


\section*{Вывод}

В данном разделе были представлены требования к разрабатываемому программному обеспечению, описание используемых типов данных, а также схемы алгоритма полного перебора и муравьиного алгоритма и классы эквивалентности для функционального тестирования.


