\chapter*{Введение}
\addcontentsline{toc}{chapter}{Введение}

Оптимизации, позволяющие улучшить работу существующих алгоритмов или помогающие решить поставленную задачу иным, более эффективным способом, были важны во все времена.
Одной из важных задач являются задачи поисков оптимальных маршрутов.
Такие задачи возможно решить полным перебором, однако данное решение является крайне неэффективным при большом числе вершин в графе расстояний (задачу поиска оптимального маршрута можно представить в виде графа --- набора вершин и рёбер).

\textbf{Цель работы:} изучение задачи коммивояжёра и решения муравьиным алгоритмом. 
Для достижения поставленной цели необходимо выполнить следующие задачи:
\begin{enumerate}[label=\arabic*)]
	\item изучить задачу коммивояжера;
	\item описать алгоритмы решения задачи коммивояжера -- полный перебор и муравьиный алгоритм;
	\item привести схемы полного перебора и муравьиного алгоритмов;
	\item описать используемые типы и структуры данных;
	\item описать структуру разрабатываемого программного обеспечения;
	\item реализовать разработанные алгоритмы;
	\item провести функциональное тестирование разработанного алгоритма;
	\item провести сравнительный анализ по времени для реализованного алгоритма;
	\item описать и обосновать полученные результаты в виде отчёта о выполненной лабораторной работе, выполненном как расчётно-пояснительная записка к работе.
\end{enumerate}