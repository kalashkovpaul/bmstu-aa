\chapter*{Введение}
\addcontentsline{toc}{chapter}{Введение}

В процессе развития компьютерных систем количество обрабатываемых данных увеличивалось, вледствие чего множество операций над наборами данных стали выполняться очень долго, поскольку чаще всего это был обычный перебор. 
Это вызвало необходимость создать новые алгоритмы, которые решают поставленную задачу на порядок быстрее стандартного решения прямого обхода. 
В том числе это касается и словарей, в которых одной из основных операций является операция поиска.


\textbf{Цель работы:} получить навык поиска по словарю при ограничении на значение признака, заданном при помощи лингвистической переменной.
Для достижения поставленной цели необходимо выполнить следующие задачи:
\begin{enumerate}[label=\arabic*)]
	\item формализовать объект и его признак;
	\item составить анкету для заполнения респондентом;
	\item провести анкетирование респондентов;
	\item построить функцию принадлежности термам числовых значений признака, описываемого лингвистической переменной, на основе статистической обработки мнений респондентов, выступающих в роли экспертов;
	\item описать 3--5 типовых вопросов на русском языке, имеющих целью запрос на поиск в словаре;
	\item описать алгоритм поиска в словаре объектов, удоволетворяющих ограничению, заданному в вопросе на ограниченном естественном языке;
	\item описать структуру данных словаря, хранящего наименование объектов согласно варианту и числовое значение признака объекта;
	\item реализовать описанный алгоритм поиска в словаре;
	\item привести примеры запросов пользователя и сформированной реализацией алгоритма поиска выборки объектов из словаря, используя составленные респондентами вопросы;
	\item дать заключение о применимости предложенного алгоритма и его ограничениях;
	\item описать и обосновать полученные результаты в виде отчёта о выполненной лабораторной работе, выполненном как расчётно-пояснительная записка к работе.
\end{enumerate}