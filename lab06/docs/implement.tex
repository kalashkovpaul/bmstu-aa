\chapter{Технологическая часть}

В данном разделе будут рассмотрены средства реализации, а также представлены листинги реализаций рассматриваемых сортировок.

\section{Средства реализации}
В данной работе для реализации был выбран язык программирования $C++$ \cite{cpp-lang}. В текущей лабораторной работе требуется замерить время работы выполняемой программы
и визуализировать результаты при помощи графиков. Инструменты для этого присутствуют в выбранном языке программирования.

Построение графиков и функциональное тестирование было осуществлено при помощи языка программирования $Python$ \cite{python-lang}, нагрузочное тестирование и замеры времени при помощи утилиты $wrk$ \cite{wrk}.

\section{Сведения о модулях программы}
Программа состоит из шести модулей:
\begin{enumerate}[label=\arabic*)]
	\item $main$ -- файл, содержащий точку входа;
	\item $Server$ -- модуль, содержащий реализацию сервера;
	\item $utils$ -- модуль, содержащий служебные алгоритмы;
	\item $constants.hpp$ -- файл, содержаший константы программы;
	\item $responses.hpp$ -- файл, содержащий шаблоны ответов. \newline
\end{enumerate}


\section{Реализации алгоритмов}

В листингах \ref{lst:server_header}, \ref{lst:server_run}, \ref{lst:server_handle} представлена реализация алгоритмов создания пула потоков, помещения запросов в очередь и обработки запроса в потоке\clearpage

\begin{center}
    \captionsetup{justification=raggedright,singlelinecheck=off}
    \begin{lstlisting}[label=lst:server_header,caption=Заголовочный файл реализации сервера, включающий в себя создание пула потоков]
class Server {
	public:
	Server() = delete;
	
	explicit Server(size_t poolSize = POOL_SIZE, int port = PORT, int countConn = CONN_COUNT)
	:poolSize(poolSize), serverPort(port), queueConnections(CONN_COUNT) {
		this->serverSocket = 0, this->clientSocket =  0;
		memset(&this->serverAddress, 0, sizeof(struct sockaddr_in));
		memset(&this->clientAddress, 0, sizeof(struct sockaddr_in));
		this->threadPool.resize(this->poolSize);
		for(size_t i = 0; i < this->poolSize; i++) {
			this->threadPool[i] = std::thread([this](){this->handleRequest();});
		}
	};
	
	int init();
	void run();
	
	private:
	int serverSocket;
	int clientSocket;
	struct sockaddr_in serverAddress;
	struct sockaddr_in clientAddress;
	int serverPort;
	
	std::vector<std::thread> threadPool;
	size_t poolSize;
	std::queue<int> requestQueue;
	
	int queueConnections;
	std::mutex queueMutex;
	std::condition_variable cv;
	
	void handleRequest();
	int loadConfig();
};
	\end{lstlisting}
\end{center}

\begin{center}
\captionsetup{justification=raggedright,singlelinecheck=off}
\begin{lstlisting}[label=lst:server_run,caption=Запуск сервера и алгоритм помещения запросов в очередь]
void Server::run() {
	while (true) {
		socklen_t addr_size = sizeof(this->clientAddress);
		this->clientSocket = accept(serverSocket, (struct sockaddr *) &this->clientAddress, &addr_size);
		char peerIP[INET_ADDRSTRLEN] = {0};
		if (inet_ntop(AF_INET, &this->clientAddress.sin_addr, peerIP, sizeof(peerIP))) {
			std::cout << "Accepted connection with " << peerIP << "\n";
		} else {
			return;
		}
		this->queueMutex.lock();
		this->requestQueue.push(clientSocket);
		this->cv.notify_one();
		this->queueMutex.unlock();
	}
}
\end{lstlisting}
\end{center}


\begin{center}
\captionsetup{justification=raggedright,singlelinecheck=off}
\begin{lstlisting}[label=lst:server_handle,caption=Алгоритм обработки запроса в потоке]
void Server::handleRequest() {
	std::unique_lock<std::mutex> lock(this->queueMutex, std::defer_lock);
	int client_socket = -1;
	while (true) {
		lock.lock();
		this->cv.wait(lock, [this]() { return !this->requestQueue.empty(); });
		client_socket = this->requestQueue.front();
		this->requestQueue.pop();
		lock.unlock();
		char req[2 * REQ_SIZE];
		recv(client_socket, req, sizeof(req), 0);
		std::string reply = handle(req);
		send(client_socket, reply.c_str(), reply.size(), 0);
		close(client_socket);
	}
}
\end{lstlisting}
\end{center}

\section{Функциональные тесты}

В таблице \ref{tbl:functional_test} приведены запросы для написанного сервера раздачи статической информации. Тесты для всех запросов пройдены \textit{успешно}.

\begin{table}[h]
\begin{center}
	\caption{\label{tbl:functional_test} Функциональные тесты}
	\begin{tabular}{|c|c|}
		\hline
		Содержимое URL & Ожидаемый результат \\ 
		\hline
		Пусто & Ответа нет\\
		Заголовок HTTP1.1  & Ответа нет\\
		/test  & Server \\
		/test/dirToTest  & /test/dirToTest/index.html \\
		/test/deep  & Статус 403\\
		/test/blablabla.txt  & Статус 403\\
		/test/deep/deeper/deepest/deep.txt & bingo, you found it\\
		/test/dirToTest/page.html/ & Статус 404\\
		/test/dirToTest/page.html?arg=val & Статус 404\\
		\text{/test/s\%20p\%20a\%20c\%20e.txt} & s p a c e.txt \\
		\text{/test/dirToTest/\%61\%2e\%68\%74\%6d\%6c} &  page.html \\
		\text{/test/wikipedia\_russia.html} & Страница Википедии \\
		/test/../../../../../../../../../../../../../etc/passwd & Статус 403 \\
		/test/text..txt & text..txt \\
%		POST: /test/dirToTest/page.html & 405 \\
%		HEAD: /test/dirToTest/page.html & 405 \\
		/test/dirToTest/page.html & page.html \\
		/test/css.css & css.css \\
		/test/js.js & js.js \\
		/test/jpg.jpg & jpg.jpg \\
		/test/jpeg.jpeg & jpeg.jpeg \\
		/test/png.png & png.png \\
		/test/gif.gif & gif.gif \\
		/test/swf.swf & swf.swf \\
		\hline
	\end{tabular}
\end{center}
\end{table}

\section*{Вывод}

В данном разделе были выбраны средства реализации, представлены листинги реализаций алгоритмов создания пула потоков, помещения запросов в очередь и обработки запроса в потоке, а также рассмотрены запросы для функционального тестирования, которое для всех запросов проходит успешно.
