\chapter{Технологическая часть}
В данном разделе будут приведены средства реализации и листинги реализованных алгоритмов.

\section{Средства реализации}
В данной работе для реализации был выбран язык программирования $Python$ \cite{pythonlang}. В текущей лабораторной работе требуется замерить процессорное время работы выполняемой программы
и визуализировать результаты при помощи графиков. Инструменты для этого присутствуют в выбранном языке программирования.

\section{Сведения о модулях программы}
Программа состоит из четырёх модулей:
\begin{enumerate}[label=\arabic*)]
	\item $main.py$ --- файл, содержащий точку входа;
	\item $utils.py$ --- файл, содержащий служебные алгоритмы;
	\item $constants.py$ --- файл, содержаший константы программы;
	\item $humans.py$ --- файл, содержащий код класса \textit{Human}, формализуемого рассматриваемый объект. 
\end{enumerate}

\section{Реализация алгоритмов}
В листинге \ref{lst:bfs} представлена реализация алгоритма поиска в словаре полным перебором.
\clearpage

\begin{center}
	\captionsetup{justification=raggedright,singlelinecheck=off}

	\begin{lstlisting}[label=lst:bfs,caption=Реализация алгоритма поиска полным перебором]
		def full_comb_search(self, key):
			k = 0
			keys = list(self.data.keys())
			for elem in keys:
				if key == elem:
					return self.data[elem]
			return -1
	\end{lstlisting}
\end{center}

\section{Функциональное тестирование}

В данном разделе будет приведена таблица с тестами (таблица \ref{table:ref1}).
\begin{center}
	\captionsetup{justification=raggedright,singlelinecheck=off}
	\begin{table}[ht]
		\centering
		\caption{Таблица тестов}
		\label{table:ref1}
		\begin{tabular}{ |c|c|c|}
			\hline
			Входные данные    & Пояснение   	  & Результат    \\ 
			\hline
			низкий			  & Средний элемент   & Ответ верный \\ \hline
			высокий 			  & Первый элемент    & Ответ верный \\ \hline
			средний 		  & Последний элемент & Ответ верный \\ \hline
			упс & Несуществующий элемент & Ответ верный (-1) \\ \hline
			123 & Несуществующий элемент & Ответ верный (-1) \\ \hline
		\end{tabular}
	\end{table}
\end{center}
Все тесты пройдены \textit{успешно}.


\section{Вывод}
В данном разделе был представлен листинг рассматриваемого алгоритма поиска в словаре, приведена информация о средствах реализации, сведения о модулях программы и было проведено функциональное тестирование.