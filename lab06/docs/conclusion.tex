\chapter*{Заключение}
\addcontentsline{toc}{chapter}{Заключение}

Поставленная цель была достигнута: изучены принципы и получены навыки организации параллельного выполнения операций на примере сервера раздачи статической информации

В ходе выполнения лабораторной работы были решены следующие задачи:

\begin{enumerate}[label=\arabic*)]
	\item были изучены основы распараллеливания вычислений;
	\item было разработатано и реализовано программное обеспечение, позволяющее раздавать статическую информацию на локальном сервере согласно паттерну thread pool;
	\item были проведены сравнение и анализ по времени обработки установленного количества поданных запросов с использованием многопоточности и без неё;
	\item подготовлен отчёт о лабораторной работе, представленный как расчётно-пояснительная записка к работе.
\end{enumerate}

Исходя из полученных результатов, при увеличении нагрузки на сервер (например, при увеличении числа открытых соединений) реализация с большим количеством потоков показала наилучший результат -- так, реализация с использованием 16 потоков одновременно оказалась более чем в 6 раз эффективнее реализации с одним потоком.
В целом, использование многопоточности показало значительный прирост количества обработанных запросов в секунду, в частности, при количестве открытых запросов большем, чем 20.

