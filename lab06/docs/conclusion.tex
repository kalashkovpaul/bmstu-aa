\chapter*{Заключение}
\addcontentsline{toc}{chapter}{Заключение}

Поставленная цель достигнута: получен навык параметризации методов на примере решения задачи коммивояжёра методом на основе муравьиного алгоритма. 

В ходе выполнения лабораторной работы были решены все задачи:

\begin{enumerate}[label=\arabic*)]
\item описана задача коммивояжёра;
\item описаны методы решения задачи коммивояжёра --- метод полного перебора и метод на основе муравьиного алгоритма;
\item приведены схемы муравьиного алгоритма и алгоритма, позволяющего решить задачу коммивояжёра методом полного перебора;
\item описаны используемые типы и структуры данных;
\item описана структура разрабатываемого программного обеспечения;
\item реализованы разработанные алгоритмы;
\item проведено функциональное тестирование разработанного алгоритма;
\item проведен сравнительный анализ выполненных реализаций по затрачиваемому времени;
\item полученные результаты описаны и обоснованы в отчёте о выполненной лабораторной работе, выполненном как расчётно-пояснительная записка к работе..
\end{enumerate}

Исходя из полученных результатов, использование муравьиного алгоритма наиболее эффективно по времени при больших размерах матриц. Так при размере матрицы, равном 2, муравьиный алгоритм медленее алгоритма полного перебора в 153 раза, а при размере матрицы, равном 9, муравьиный алгоритм быстрее алгоритма полного перебора в раз, а при размере в 10 -- уже в 21 раз. Следовательно, при размерах матриц больше 8 следует использовать муравьиный алгоритм, но стоит учитывать, что он не гарантирует оптимального решения, в отличие от метода полного перебора.

Даны рекомендации о применимости метода на основании муравьиного алгоритма к решению задачи коммивояжёра. 
