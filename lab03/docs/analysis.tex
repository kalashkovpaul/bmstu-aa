\chapter{Аналитическая часть}
В этом разделе будут рассмотрены алгоритмы сортировок -- блинная, поразрядная, бинарным деревом.

\section{Блинная сортировка}
\textbf{Сортировка вставками} \cite{pancake-sort} \text{(англ. \textit{pancake sort})}-- алгоритм сортировки.
Единственная операция, допустимая в алгоритме -- переворот элементов последовательности до какого-либо индекса. В отличие от традиционных алгоритмов, в которых минимизируют количество сравнений, в блинной сортировке требуется сделать как можно меньше переворотов. Процесс можно визуально представить как стопку блинов, которую тасуют путём взятия нескольких блинов сверху и их переворачивания.

В данной работе будет рассмотрен вариант блинной сортировки, основанный на сортировке выбором.
Алгоритм состоит из нескольких шагов:
\begin{enumerate}
	\item Найти номер максимального числа в неотсортированной части массива.
	\item Произвести ``переворот`` неотсортированной части так, чтобы максимум встал на своё место.
	\item Аналогично сортировать остаточную часть массива, исключив из рассмотрения уже отсортированные элементы.
\end{enumerate}

\section{Поразрядная сортировка}
\textbf{Поразрядная сортировка} \cite{radix-sort} \text{(англ. \textit{radix sort})} -- сортировка, исходно предназначенная для сортировки целых чисел, записанных цифрами. 
Поскольку в памяти компьютеров любая информация записывается целыми числами, алгоритм пригоден для сортировки любых объектов, запись которых можно поделить на ``разряды``, содержащие сравнимые значения. 
Таким образом можно сортировать не только числа, записанные в виде набора цифр, но и строки, являющиеся набором символов, и даже произвольные значения в памяти, представленные в виде набора байт.


Сравнение производится поразрядно: сначала сравниваются значения одного крайнего разряда и элементы группируются по результатам этого сравнения, затем сравниваются значения следующего разряда, соседнего, и элементы либо упорядочиваются по результатам сравнения значений этого разряда внутри образованных на предыдущем проходе групп, либо переупорядочиваются в целом, но сохраняя относительный порядок, достигнутый при предыдущей сортировке. Затем аналогично делается для следующего разряда, и так до конца.


Для поразрядной сортировки целых чисел в рамках одного разряда удобно использовать сортировку подсчётом -- рассмотрим её ниже.

\subsection{Сортировка подсчётом}

\textbf{Сортировка подсчётом} \cite{counting-sort} \text{(англ. \textit{counting sort})} --  алгоритм сортировки, в котором используется диапазон чисел сортируемого массива (списка) для подсчёта совпадающих элементов. Данный алгоритм состоит из двух основных этапов:
\begin{enumerate}
	\item Создать вспомогательный массив, состоящий из нулей, затем последовательно прочитать элементы входного массива, для каждого из них увеличить соответствующее значение вспомогательного массива на единицу.
	\item Пройтись по вспомогательному массиву и для i-го значения X записать в результирующий массив i-ое значение X раз
\end{enumerate}

\section{Сортировка бинарным деревом}
\textbf{Сортировка бинарным деревом} \cite{bst-sort} \text{(англ. \textit{binary search tree sort})} -- универсальный алгоритм сортировки, заключающийся в построении двоичного дерева поиска по элементам массива (списка), с последующей сборкой результирующего массива путём обхода узлов построенного дерева в необходимом порядке следования ключей.
	

\textbf{Бинарное дерево поиска } \cite{bst-sort} \text{(англ. \textit{binary search tree})} -- двоичное дерево, для которого выполняются следующие дополнительные условия:
\begin{enumerate}
	\item Оба поддерева -- левое и правое -- являются двоичными деревьями поиска.
	\item У всех узлов левого поддерева произвольного узла X значения ключей данных меньше, нежели значение ключа данных самого узла X;
	\item У всех узлов правого поддерева произвольного узла X значения ключей данных больше либо равны, нежели значение ключа данных самого узла X.
\end{enumerate}


Преобразование бинарного дерева поиска в результирующий массив с отсортированными данными следует делать при помощи обратного обхода бинарного дерева -- обхода, при котором сначала обходится левое поддерево данной вершины, затем правое, затем данная вершина.


\section*{Вывод}
В данной работе необходимо реализовать алгоритмы сортировки, описанные в данном разделе, а также провести их теоритическую оценку и проверить ее экспериментально.