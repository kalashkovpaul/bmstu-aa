\chapter{Аналитическая часть}
В этом разделе будут рассмотрены алгоритмы сортировок -- блинная, поразрядная, бинарным деревом.

\section{Блинная сортировка}
\textbf{Сортировка вставками} \cite{pancake-sort} -- алгоритм сортировки.
Единственная операция, допустимая в алгоритме -- переворот элементов последовательности до какого-либо индекса. В отличие от традиционных алгоритмов, в которых минимизируют количество сравнений, в блинной сортировке требуется сделать как можно меньше переворотов. Процесс можно визуально представить как стопку блинов, которую тасуют путём взятия нескольких блинов сверху и их переворачивания.

В данной работе будет рассмотрен вариант блинной сортировки, основанный на сортировке выбором.
Алгоритм состоит из нескольких шагов:
\begin{enumerate}
	\item Найти номер максимального числа в неотсортированной части массива.
	\item Произвести ``переворот`` неотсортированной части так, чтобы максимум встал на своё место.
	\item Аналогично сортировать остаточную часть массива, исключив из рассмотрения уже отсортированные элементы.
\end{enumerate}

\section{Сортировка перемешиванием}
\textbf{Сортировка перемешиванием} \cite{sheyker-sort} - сортировка, которая является модификацией сортировки пузырьком. Различие состоит в том, что в рамках одной итерации происходит проход по массиву в обоих направлениях. В сортировке пузырьком просходит только проход слева-направо, то есть в одном направлении.

Суть сортировки - сначала идет обычный проход слева-направо, как при обычном пузырьке. Затем, начиная с элемента, который находится перед последним отсортированным, начинается проход в обратном направлении. Здесь тикже сравниваются элементы меняются местами при необходимости.


\section{Гномья сортировка}
\textbf{Гномья сортировка} \cite{gnomme-sort} - алгоритм сортировки, который использует только один цикл, что является редкостью.

В этой сортировке массив просматривается селва-направо, при этом сравниваются и, если нужно, меняются соседние элементы. Если происходит обмен элементов, то происходит возвращение на один шаг назад. Если обмена не было - агоритм продолжает просмотр массива в поисках неупорядоченных пар.

\section*{Вывод}
В данной работе необходимо реализовать алгоритмы сортировки, описанные в данном разделе, а также провести их теоритическую оценку и проверить ее экспериментально.