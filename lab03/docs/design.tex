\chapter{Конструкторская часть}
В данном разделе будут рассмотрены схемы алгоритмов сортировок (блинной, поразрядной, бинарным деревом), а также найдена их трудоёмкость


\section{Разработка алгоритмов}
На рисунках 2.1, 2.2, 2.3 и 2.4 представлены схемы алгоритмов сортировки -- блинной, поразрядной, подсчётом и бинарным деревом.
\img{220mm}{pancakeSort.png}{Схема алгоритма блинной сортировки}
\img{220mm}{radixSort.png}{Схема алгоритма поразрядной сортировки}
\img{220mm}{countingSort.png}{Схема алгоритма сортировки подсчётом}
\img{220mm}{bstSort.png}{Схема алгоритма сортировки бинарным деревом}

\clearpage

\section{Модель вычислений для проведения оценки трудоёмкости}
Введем модель вычислений \cite{model}, которая потребуется для определния трудоемкости каждого отдельно взятого алгоритма сортировки.

\begin{enumerate}[label=\arabic*)]
	\item Операции из списка (\ref{for:operations}) имеют трудоёмкость, равную 1:
	\begin{equation}
		\label{for:operations}
		+, -, /, *, \%, =, +=, -=, *=, ==, !=, <, >, <=, >=, [], ++, {-}-
	\end{equation}
	\item Трудоёмкость оператора выбора \code{if условие then A else B} рассчитывается как
	\begin{equation}
		\label{for:if}
		f_{if} = f_{\text{условия}} +
		\begin{cases}
			f_A, & \text{если условие выполняется,}\\
			f_B, & \text{иначе.}
		\end{cases}
	\end{equation}
	\item Трудоёмкость цикла рассчитывается как
	\begin{equation}
		\label{for:cycle}
		f_{for} = f_{\text{инициализации}} + f_{\text{сравнения}} + N(f_{\text{тела}} + f_{\text{инкремент}} + f_{\text{сравнения}}).
	\end{equation}
	\item Трудоёмкость вызова функции равна 0.
\end{enumerate}


\section{Трудоёмкость алгоритмов}

Определим трудоёмкость выбранных алгоритмов сортировки по схемам алгоритмов.

\subsection{Алгоритм блинной сортировки}

Трудоёмкость алгоритма блинной сортировки включает в себя приведённые ниже составляющие:
\begin{enumerate}
	\item Трудоёмкость сравнения внешнего цикла $if  (size > 1)$, которая равна (\ref{for:outer_if}):
	\begin{equation}
		\label{for:outer_if}
		f_{if} =1 + \begin{cases}
			0, & \text{л.с.}\\
			f_{cycle}, & \text{х.с.}\\
		\end{cases}
	\end{equation}
	\item Трудоёмкость цикла, количество итераций которого равна  $size$, которая равна (\ref{for:outer_cycle}):
	\begin{equation}
		\label{for:outer_cycle}
		f_{cycle} = 2 + size \cdot f_{body}
	\end{equation}
	\item Трудоёмкость тела цикла $f_{body}$ равна (\ref{for:outer_cycle_body}):
	\begin{equation}
		\label{for:outer_cycle_body}
		f_{body} = 1 + \frac{(size + 2) \cdot 5}{2} + 3 + f_{if} + 2
	\end{equation}
	\item Трудоёмкость условия во внутреннем цикле, которая равна (\ref{for:outer_cycle_body_if}):
	\begin{equation}
		\label{for:outer_cycle_body_if}
		f_{if} = 4 + \begin{cases}
			2, & \text{л.с.}\\
			7 + \frac{size \cdot 3 \cdot 2}{2} , & \text{х.с.}\\
		\end{cases}
	\end{equation}
\end{enumerate}

Трудоёмкость алгоритма в лучшем случае (массив уже отсортирован)(\ref{for:pancake_best}):
\begin{align}
	\label{for:pancake_best}
	f_{best} &= 2 + size \cdot (6 + \frac{5 \cdot (size + 2)}{2} + 2)  \\
	&= 2 + \frac{23}{2} \cdot size + \frac{5\cdot size^2}{2} \approx \frac{5\cdot size^2}{2} \approx O(size^2)
\end{align}

Трудоёмкость в худшем случае (\ref{for:pancake_worst}):
\begin{align}
	\label{for:pancake_worst}
	f_{worst} &= 2 + size \cdot (6 + \frac{5 \cdot (size + 2)}{2} + 7 + size \cdot 3)  \\
	&= 2 + \frac{39}{2} \cdot size + \frac{7\cdot size^2}{2} \approx \frac{7\cdot size^2}{2} \approx O(size^2)
\end{align}


\subsection{Алгоритм поразрядной сортировки}

Трудоёмкость алгоритма поразрядной сортировки равна %(\ref{for:radix_sort_perf}):
\begin{align}
	\label{for:radix_sort_perf}
    f_{radix} = 1 + 2 + 5 \cdot size + 1 + 2 + m * (f_{count} + 1 + 2),
\end{align}
где \\
$m$ - количество разрядов в максимальном элементе \\
$f_{count}$ - трудоёмкость алгоритма сортировки подсчётом.


Трудоёмкость алгоритма сортировки подсчётом (\ref{for:count_sort_perf}):
\begin{align}
\begin{split}
	\label{for:count_sort_perf}
    f_{count} &= size + 10  + 2 + size \cdot 8 + 2 + 10 \cdot 6 + 3 \\
    &+ size \cdot 14 + 1 + size \cdot 5 \\
    &= 78 + 28 \cdot size
\end{split}
\end{align}

Итоговая трудоёмкость порязрядной сортировки, использующей сортировку подсчётом в рамках одного разряда: %(\ref{for:total_radix})
\begin{align}
\begin{split}
	\label{for:total_radix}
	f_{radix} &= 6 + 2 \cdot size + m \cdot (28 \cdot size + 78) \approx 28 \cdot m \cdot size \approx O(size\cdot m)
\end{split}
\end{align}

\subsection{Алгоритм сортировки бинарным деревом}
Трудоёмкость данного алгоритма посчитаем следующим образом: сортировка -- преобразование массива или списка в бинарное дерево поиска посредством операции вставки нового элемента в бинарное дерево.
Операция вставки в бинарное дерево имеет сложность $ \log_{2}(size)$, где $size$ - количество элементов в дереве.
Для преобразования массива или списка размером $size$ потребуется использовать операцию вставки в биннарное дерево $size$ раз, таким образом, итоговая трудоёмкость данной сортировки будет равна (\ref{for:binary_sort_perf}):
\begin{align}
\begin{split}
	\label{for:binary_sort_perf}
	f_{radix} &= size \cdot \log_{2}(size)
\end{split}
\end{align}


\section*{Вывод}

Были разработаны схемы всех трех алгоритмов сортировки. Также для каждого из них были рассчитаны трудоёмкости по введённой модели вычислений с учётом лучших и худших случаев.
