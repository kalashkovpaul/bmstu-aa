\chapter*{Заключение}
\addcontentsline{toc}{chapter}{Заключение}

Была достигнута цель работы: изучены и исследованы трудоёмкости алгоритмов сортировок -- блинной, поразрядной, бинарным деревом. 
Также в ходе выполнения лабораторной работы были решены следующие задачи:

\begin{enumerate}[label=\arabic*)]
	\item были изучены и реализованы алгоритмы сортировки: блинная, поразрядная и бинарным деревом;
	\item было измерено время работы реализаций алгоритмов выбранных сортировок;
    \item были проведены сравнение и анализ трудоёмкостей алгоритмов на основе теоретических расчетов;
    \item были проведены сравнение и анализ реализаций алгоритмов по затраченным процессорному времени и памяти;
	\item был подготовлен отчёт о лабораторной работе, представленный как расчётно-пояснительная записка к работе.
\end{enumerate}



Исходя из полученных результатов, сортировка бинарным деревом на отсортированных массивах и блинная сортировка на случайном массиве работают дольше всех (на длине массива в 800 элементов примерно в 40 раз дольше, чем поразрядная сортировка), при этом поразрядная сортировка показала себя лучше всех на любых данных. Можно сделать вывод, что использование сортировки бинарным деревом показывает наилучший результат при случайных, никак не отсортированных данных, т.к. при отсортированных данных обычное бинарное дерево вырождается в связный список, из-за чего вырастает высота дерева. Поразрядная сортировка же эффективнее в том случае, когда приблизительно известно максимальное количество разрядов в сортируемых данных.
