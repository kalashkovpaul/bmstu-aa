\chapter*{Введение}
\addcontentsline{toc}{chapter}{Введение}

Сортировка -- перегруппировка некой последовательности или кортежа в определённом порядке. 
Это одна из главных процедур обработки структурированных данных. 
Расположение элементов в определённом порядке позволяет более эффективно проводить работу с последовательностью данных, в частности, при поиске некоторых данных. \newline

Существует множество алгоритмов сортировки, но любой алгоритм сортировки выполняет следующие шаги:
\begin{enumerate}[label=\arabic*)]
	\item сравнение, которое определяет, как упорядочена пара элементов;
    \item перестановка для изменения порядка элементов;
    \item последовательность действий, использующую сравнения и перестановки.
\end{enumerate}

Что касается самого поиска, то при работе с отсортированным набором данных время, которое нужно на нахождение элемента, пропорционально логарифму количества элементов, если использовать бинарный поиск. Последовательность, данные которой расположены в хаотичном порядке, занимает время, которое пропорционально количеству элементов, что куда больше логарифма.

\textbf{Цель работы:} изучение и исследование трудоёмкости алгоритмов сортировок: блинной, поразрядной, бинарным деревом.

\textbf{Задачи работы:}
\begin{enumerate}[label=\arabic*)]
	\item изучить и реализовать алгоритмы сортировки -- блинной, поразрядной, бинарным деревом.
    \item измерить временя работы реализаций алгоритмов выбранных сортировок.
    \item сравнить и проанализировать трудоёмкости алгоритмов на основе теоретических расчетов.
    \item сравнить и проанализировать реализации алгоритмов по затраченным процессорному времени и памяти;
	\item описать и обосновать полученные результаты в отчёте о выполненной лабораторной работе, выполненном как расчётно-пояснительная записка к работе.
\end{enumerate}