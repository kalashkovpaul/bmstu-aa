\chapter*{Заключение}
\addcontentsline{toc}{chapter}{Заключение}

Цель, которая была поставлена в начале лабораторной работы была достигнута, а также в ходе выполнения лабораторной работы были решены следующие задачи:

\begin{enumerate}[label=\arabic*)]
	\item были изучены и реализованы алгоритмы сортировки: блинная, поразрядная и бинарным деревом;
	\item была выбрана модель вычисления и проведен сравнительный анализ трудоёмкостей выбранных алгоритмов сортировки;
    \item на основе экспериментальных данных проведено сравнение выбранных алгоритмов сортировки;
	\item подготовлен отчёт о лабораторной работе, представленный как расчётно-пояснительная записка к работе.
\end{enumerate}

Исходя из полученных результатов, сортировка бинарным деревом на отсортированных массивах и блинная сортировка на случайном массиве работают дольше всех (примерно в 40 раза дольше, чем поразрядная сортировка), при этом поразрядная сортировка показала себя лучше всех на любых данных. Можно сделать вывод, что использование сортировки бинарным деревом показывает наилучший результат при случайных, никак не отсортированных данных, т.к. при отсортированных данных обычное бинарное дерево вырождается в связный список, из-за чего вырастает высота дерева. Поразрядная сортировка же эффективнее в том случае, когда приблизительно известно максимальное количество разрядов в сортируемых данных.