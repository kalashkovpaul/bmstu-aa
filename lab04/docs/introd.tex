\chapter*{Введение}
\addcontentsline{toc}{chapter}{Введение}

По мере развития вычислительных систем программисты столкнулись с необходимостью производить параллельную обработку данных для улучшения отзывчивости системы, ускорения производимых вычислений и рационального использования вычислитлельных мощностей. Благодаря развитию процессоров стало возможным использовать один процессор для выполнения нескольких параллельных операций, что дало начало термину ``многопоточность``.

\textbf{Цель работы:} изучение принципов и получение навыков организации  параллельного выполнения операций на примере сервера раздачи статической информации. Для достижения поставленной цели необходимо выполнить следующие задачи:
\begin{enumerate}[label=\arabic*)]
	\item изучить основы распараллеливания вычислений;
	\item разработать и реализовать программное обеспечение, позволяющее раздавать статическую информацию на локальном сервере согласно паттерну thread pool;
	\item сравить и проанализировать по времени обработки установленного количества поданных запросов с использованием многопоточности и без неё;
	\item описать и обосновать полученные результаты в виде отчёта о выполненной лабораторной работе, выполненном как расчётно-пояснительная записка к работе.
\end{enumerate}