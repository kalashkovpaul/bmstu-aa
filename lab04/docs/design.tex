\chapter{Конструкторская часть}
В данном разделе будут рассмотрены схема алгоритмов, реализующих сервер раздачи статической информации согласно паттерну thread pool с распараллеливанием, а также без него.

\section{Требования к ПО}
Ряд требований к программе:
\begin{itemize}
	\item сервер должен работать на локальном доменном имени на порту 8080;
    \item обращение к серверу происходит посредством браузера, запросы HTTP1.1;
    \item в ответ должен отдаваться статический файл, указанный в предыдущем пункте.\newline
\end{itemize}

\section{Разработка алгоритмов}
На рисунках 2.1, 2.2 представлены схемы алгоритмов помещения запроса в очередь, а также обработки запроса в потоке.
Заметим, что реализация сервера согласно данным схемам будет одинаковой как в случае исполльзовании нескольких потоков, так и в случае отсутствия параллельности, в силу того, что даже непараллельная реализация является реализацией с использованием одного потока.
\img{240mm}{server.png}{Схема алгоритма сервера помещения запроса в очередь}
\img{240mm}{handle.png}{Схема алгоритма обработки запроса в потоке}

\clearpage

\section*{Вывод}

Были разработаны схемы всех алгоритмов помещения запроса в очередь, а также обработки и получения необходимого запроса из очередь в потоке. Критическая секция была снижена настолько, насколько было возможно, что позволит наиболее эффективно использовать многопоточность.
