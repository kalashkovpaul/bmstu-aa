\chapter{Исследовательская часть}

В данном разделе будут приведён пример работы программа, а также проведён сравнительный анализ алгоритмов при различных ситуациях на основе полученных данных.

\section{Технические характеристики}

Технические характеристики устройства, на котором выполнялось тестирование представлены далее:

\begin{itemize}
	\item операционная система: Mac OS Monterey Версия 12.5.1 (21G83) \cite{macos} x86\_64;
	\item память: 16 GB;
	\item процессор: 2,7 GHz 4‑ядерный процессор Intel Core i7 \cite{intel}.
\end{itemize}

При тестировании ноутбук был включен в сеть электропитания. Во время тестирования ноутбук был нагружен только встроенными приложениями окружения, а также системой тестирования.

\section{Демонстрация работы программы}

На рисунке \ref{img:example} представлен результат работы программы.

\img{100mm}{example}{Пример работы программы}
\clearpage

\section{Время выполнения алгоритмов}

Как было сказано выше, используется функция замера процессорного времени process\_time(...) из библиотеки time на Python. Функция возвращает пользовательское процессорное временя типа float.

Использовать функцию приходится дважды, затем из конечного времени нужно вычесть начальное, чтобы получить результат.

Результаты замеров времени работы алгоритмов сортировки на различных входных данных (в мс) приведены в таблицах \ref{tbl:best}, \ref{tbl:worth} и \ref{tbl:random}.

\begin{table}[h]
	\begin{center}
		\begin{threeparttable}
		\captionsetup{justification=raggedleft,singlelinecheck=off}
		\caption{Отсортированные данные}
		\label{tbl:best}
		\begin{tabular}{|c|c|c|c|}
			\hline
			Размер & Блинная &  Поразрядная &  Бинарным деревом \\
			\hline
			100 & 0.1662 & 0.0714 & 0.8730 \\ 
			\hline
			200 & 0.5113 & 0.2058 & 3.3267 \\ 
			\hline
			300 & 1.1026 & 0.3131 & 7.6354 \\ 
			\hline
			400 & 2.0140 & 0.4364 & 13.6751 \\ 
			\hline
			500 & 3.3046 & 0.5591 & 21.5524 \\ 
			\hline
			600 & 5.0567 & 0.6798 & 31.3052 \\ 
			\hline
			700 & 6.6944 & 0.7852 & 43.0406 \\ 
			\hline
			800 & 8.5163 & 0.8766 & 56.4318 \\ 
			\hline
		\end{tabular}
		\end{threeparttable}
    \end{center}
\end{table}


\begin{table}[h]
	\begin{center}
		\begin{threeparttable}
		\captionsetup{justification=raggedleft,singlelinecheck=off}
		\caption{Отсортированные в обратном порядке данные}
		\label{tbl:worth}
		\begin{tabular}{|c|c|c|c|}
			\hline
			Размер & Блинная &  Поразрядная &  Бинарным деревом \\
			\hline
			100 & 0.1606 & 0.1048 & 0.7138 \\ 
			\hline
			200 & 0.5005 & 0.2008 & 2.7633 \\ 
			\hline
			300 & 1.0747 & 0.3110 & 6.3060 \\ 
			\hline
			400 & 1.9383 & 0.4312 & 11.3831 \\ 
			\hline
			500 & 3.1148 & 0.5427 & 18.0577 \\ 
			\hline
			600 & 4.6409 & 0.6693 & 26.0260 \\ 
			\hline
			700 & 6.7969 & 0.8317 & 36.7397 \\ 
			\hline
			800 & 8.7922 & 0.9583 & 47.2628 \\ 
			\hline
		\end{tabular}
		\end{threeparttable}
    \end{center}
\end{table}


\begin{table}[h]
	\begin{center}
		\begin{threeparttable}
		\captionsetup{justification=raggedleft,singlelinecheck=off}
		\caption{Случайные данные}
		\label{tbl:random}
		\begin{tabular}{|c|c|c|c|}
			\hline
			 Размер & Блинная &  Поразрядная &  Бинарным деревом \\
			\hline
			100 & 0.2734 & 0.1043 & 0.1560 \\ 
			\hline
			200 & 0.8321 & 0.2090 & 0.3756 \\ 
			\hline
			300 & 1.6837 & 0.3142 & 0.6025 \\ 
			\hline
			400 & 2.8938 & 0.4281 & 0.9785 \\ 
			\hline
			500 & 4.4438 & 0.5419 & 1.1784 \\ 
			\hline
			600 & 6.4153 & 0.6704 & 1.5523 \\ 
			\hline
			700 & 8.6692 & 0.7678 & 1.9018 \\ 
			\hline
			800 & 11.3752 & 0.8992 & 2.2986 \\ 
			\hline
		\end{tabular}
		\end{threeparttable}
    \end{center}
\end{table}


Также на рисунках \ref{img:graph_sorted}, \ref{img:graph_sorted_back}, \ref{img:graph_random} приведены графические результаты замеров работы сортировок в зависимости от размера входного массива.


\img{100mm}{graph_sorted}{Отсортированный массив}
\img{100mm}{graph_sorted_back}{Отсортированный в обратном порядке массив}
\img{100mm}{graph_random}{Случайный массив}
\clearpage

\section*{Вывод}
Исходя из полученных результатов, сортировка бинарным деревом на отсортированных массивах и блинная сортировка на случайном массиве работают дольше всех (примерно в 40 раза дольше, чем поразрядная сортировка), при этом поразрядная сортировка показала себя лучше всех на любых данных. Можно сделать вывод, что использование сортировки бинарным деревом показывает наилучший результат при случайных, никак не отсортированных данных, т.к. при отсортированных данных обычное бинарное дерево вырождается в связный список, из-за чего вырастает высота дерева. Поразрядная сортировка же эффективнее в том случае, когда приблизительно известно максимальное количество разрядов в сортируемых данных.

Теоретические результаты замеров и полученные практически результаты совпадают.