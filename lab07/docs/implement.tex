\chapter{Технологическая часть}

В данном разделе будут рассмотрены средства реализации, а также представлены листинги алгоритма, осуществляющего полный перебор путей, и муравьиного алгоритма.

\section{Средства реализации}
В данной работе для реализации был выбран язык программирования $Python$ \cite{python-lang}. В текущей лабораторной работе требуется замерить процессорное время работы выполняемой программы
и визуализировать результаты при помощи графиков. Инструменты для этого присутствуют в выбранном языке программирования.

Время работы было замерено с помощью функции \textit{process\_time(...)} из библиотеки $time$ \cite{python-lang-time}.

\section{Сведения о модулях программы}
Программа состоит из шести модулей:
\begin{enumerate}[label=\arabic*)]
	\item $main.py$ --- файл, содержащий точку входа;
	\item $menu.py$ --- файл, содержащий код меню программы;
	\item $test.py$ --- файл, содержаший код тестирования алгоритмов;
	\item $utils.py$ --- файл, содержащий служебные алгоритмы;
	\item $constants.py$ --- файл, содержаший константы программы;
	\item $algorythms.py$ --- файл, содержащий код всех алгоритмов. 
\end{enumerate}


\section{Реализации алгоритмов}

В листинге \ref{lst:full_comb} представлен алгоритм полного перебора путей, а в листингах \ref{lst:ant_alg}--\ref{lst:upd_pher} --- муравьиный алгоритм и дополнительные к нему функции.

\clearpage

\begin{center}
	\captionsetup{justification=raggedright,singlelinecheck=off}
	\begin{lstlisting}[label=lst:full_comb,caption=Реализация алгоритма полного перебора путей]
		def fullCombinationAlg(matrix, size):
			places = np.arange(size)
			placesCombinations = list()
			
			for combination in it.permutations(places):
				combArr = list(combination)
				placesCombinations.append(combArr)
			
			minDist = float("inf")
			
			for i in range(len(placesCombinations)):
				placesCombinations[i].append(\
				placesCombinations[i][0])
				curDist = 0
				for j in range(size):
					startCity = placesCombinations[i][j]
					endCity = placesCombinations[i][j + 1]
					curDist += matrix[startCity][endCity]
				
				if (curDist < minDist):
					minDist = curDist
					bestWay = placesCombinations[i]
			
			return minDist, bestWay
	\end{lstlisting}
\end{center}

\clearpage

\begin{center}
	\captionsetup{justification=raggedright,singlelinecheck=off}
	\begin{lstlisting}[label=lst:ant_alg,caption=Реализация муравьиного алгоритма]
	def antAlgorithm(matrix, places, alpha, beta, k_evaporation, days):
		q = calcQ(matrix, places)
		bestWay = []
		minDist = float("inf")
		pheromones = calcPheromones(places)
		visibility = calcVisibility(matrix, places)
		ants = places
		for day in range(days):
			route = np.arange(places)
			visited = calcVisitedPlaces(route, ants)
			for ant in range(ants):
				while (len(visited[ant]) != ants):
					pk = findWays(pheromones, visibility, visited, places, ant, alpha, beta)
					chosenPlace = chooseNextPlaceByPosibility(pk)
					visited[ant].append(chosenPlace - 1)
				
				visited[ant].append(visited[ant][0])
				
				curLength = calcLength(matrix, visited[ant])
				
				if (curLength < minDist):
					minDist = curLength
					bestWay = visited[ant]
				
			pheromones = updatePheromones(matrix, places, visited, pheromones, q, k_evaporation)
		
		return minDist, bestWay
	\end{lstlisting}
\end{center}


\clearpage


\begin{center}
	\captionsetup{justification=raggedright,singlelinecheck=off}
	\begin{lstlisting}[label=lst:prob_arr,caption=Реализация алгоритма нахождения массива вероятностей переходов в непосещенные города]
def findWays(pheromones, visibility, visited, places, ant, alpha, beta):
	pk = [0] * places
	
	for place in range(places):
		if place not in visited[ant]:
			ant_place = visited[ant][-1]
			pk[place] = pow(pheromones[ant_place][place], alpha) * \
			pow(visibility[ant_place][place], beta)
		else:
			pk[place] = 0
	
	sum_pk = sum(pk)
	
	for place in range(places):
		pk[place] /= sum_pk
	
	return pk
	\end{lstlisting}
\end{center}

\begin{center}
	\captionsetup{justification=raggedright,singlelinecheck=off}
	\begin{lstlisting}[label=lst:pheromone_above_Zero,caption=Реализация алгоритма нахождения массива вероятностей переходов в непосещенные города]
def calcPheromones(size):
min_phero = 1
pheromones = [[min_phero for i in range(size)] for j in range(size)]
return pheromones
	\end{lstlisting}
\end{center}


\clearpage

\begin{center}
	\captionsetup{justification=raggedright,singlelinecheck=off}
	\begin{lstlisting}[label=lst:choose_next,caption=Реализация алгоритма выбора следующего города]
def chooseNextPlaceByPosibility(pk):
	posibility = random()
	choice = 0
	chosenPlace = 0
	
	while ((choice < posibility) and (chosenPlace < len(pk))):
		choice += pk[chosenPlace]
		chosenPlace += 1
	
	return chosenPlace
	\end{lstlisting}
\end{center}


\begin{center}
	\captionsetup{justification=raggedright,singlelinecheck=off}
	\begin{lstlisting}[label=lst:upd_pher,caption=Реализация алгоритма обновления матрицы феромонов]
	def updatePheromones(matrix, places, visited, pheromones, q, k_evaporation):
		ants = places
		
		for i in range(places):
			for j in range(places):
				delta = 0
				for ant in range(ants):
					length = calcLength(matrix, visited[ant])
					delta += q / length
				
				pheromones[i][j] *= (1 - k_evaporation)
				pheromones[i][j] += delta
				if (pheromones[i][j] < MIN_PHEROMONE):
					pheromones[i][j] = MIN_PHEROMONE
		
		return pheromones
	\end{lstlisting}
\end{center}

\section{Функциональные тесты}

В таблице \ref{tbl:functional_test} приведены тесты для функций программы. Все функциональные тесты пройдены \textit{успешно}.

\begin{center}
	\captionsetup{justification=raggedright,singlelinecheck=off}
	\begin{longtable}[c]{|c|c|c|c|c|}
		\caption{Функциональные тесты\label{tbl:functional_test}} \\ \hline
		Матрица смежности & Ожидаемый результат & Результат программы \\
		\hline
		$ \begin{pmatrix}
			0 &  4 &  2 &  1 & 7 \\
			4 &  0 &  3 &  7 & 2 \\
			2 &  3 &  0 & 10 & 3 \\
			1 &  7 & 10 &  0 & 9 \\
			7 &  2 &  3 &  9 & 0
		\end{pmatrix}$ &
		15, [0, 2, 4, 1, 3, 0] &
		15, [0, 2, 4, 1, 3, 0] \\
		
		$ \begin{pmatrix}
			0 & 1 & 2 \\
			1 & 0 & 1 \\
			2 & 1 & 0	
		\end{pmatrix}$ &
		4, [0, 1, 2, 0] &
		4, [0, 1, 2, 0] \\
		
		$ \begin{pmatrix}
			0 & 15 & 19 & 20 \\
			15 &  0 & 12 & 13 \\
			19 & 12 &  0 & 17 \\
			20 & 13 & 17 &  0
		\end{pmatrix}$ &
		64, [0, 1, 2, 3, 0] &
		64, [0, 1, 2, 3, 0] \\
		\hline
	\end{longtable}
\end{center}

\section{Вывод}

Были представлены листинги всех реализаций алгоритмов --- полного перебора и муравьиного. Также в данном разделе была приведена информации о выбранных средствах для разработки алгоритмов и сведения о модулях программы, проведено функциональное тестирование.
