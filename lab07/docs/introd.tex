\chapter*{Введение}
\addcontentsline{toc}{chapter}{Введение}

Оптимизации, позволяющие улучшить работу существующих алгоритмов или помогающие решить поставленную задачу иным, более эффективным способом, были важны во все времена.
Одной из важных задач являются задачи поисков оптимальных маршрутов.
Такие задачи возможно решить полным перебором, однако данное решение является крайне неэффективным при большом числе вершин в графе (задачу поиска оптимального маршрута можно представить в виде графа --- набора вершин и рёбер).

\textbf{Цель работы:} получение навыка параметризации методов на примере решения задачи коммивояжёра методом на основе муравьиного алгоритма. 
Для достижения поставленной цели необходимо выполнить следующие задачи:
\begin{enumerate}[label=\arabic*)]
	\item описать задачу коммивояжёра;
	\item описать методы решения задачи коммивояжёра --- метод полного перебора и метод на основе муравьиного алгоритма;
	\item привести схемы муравьиного алгоритма и алгоритма, позволяющего решить задачу коммивояжёра методом полного перебора;
	\item описать используемые типы и структуры данных;
	\item описать структуру разрабатываемого программного обеспечения;
	\item реализовать разработанные алгоритмы;
	\item провести функциональное тестирование разработанного алгоритма;
	\item провести сравнительный анализ выполненных реализаций по затрачиваемому времени;
	\item описать и обосновать полученные результаты в виде отчёта о выполненной лабораторной работе, выполненном как расчётно-пояснительная записка к работе.
\end{enumerate}