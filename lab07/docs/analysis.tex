\chapter{Аналитическая часть}
В этом разделе будет представлена информация о задаче коммивояжёра, а также о способах её решения  --- методе полного перебора и методе на основе муравьиного алгоритма.

\section{Задача коммивояжёра}
\textbf{Задача коммивояжёра} \text{(англ. \textit{traveling salesman problem})} --- (задача о бродячем торговце) одна из самых важных задач всей транспортной логистики, в которой рассматриваются вершины графа, а также матрица смежности (для расстояния между вершинами)\,\cite{task}. 
Задача заключается в том, чтобы найти такой порядок посещения вершин графа, при котором путь будет минимален, каждая вершина будет посещена лишь один раз, а возврат произойдет в начальную вершину. 

\section{Метод полного перебора для решения задачи коммивояжёра}

\textbf{Полный перебор для задачи коммивояжёра}~\cite{full-comb} имеет высокую сложность алгоритма ($n!$), где $n$ --- количество городов. 
Суть в полном переборе всех возможных путей в графе и выбор наименьшего из них. 
Решение будет получено, но имеются большие затраты по времени выполнения при уже небольшом количестве вершин в графе.


\section{Метод на основе муравьиного алгоритма}

\textbf{Муравьиный алгоритм} (англ. \textit{ant colony optimization})  \cite{full-comb} --- метод решения задачи оптимизации, основаный на принципе поведения колонии муравьев.

Муравьи действуют, руководствуясь органами чувств. 
Каждый муравей оставляет на своём пути феромоны, чтобы другие могли ориентироваться. 
При большом количестве муравьев наибольшее количество феромона остаётся на наиболее посещаемом пути, посещаемость же может быть связана с длинами рёбер.

Суть в том, что отдельно взятый муравей мало что может, поскольку он способен выполнять только максимально простые задачи. Но при большом числе других таких муравьев они могут выступать самостоятельными вычислительными единицами. Муравьи используют непрямой обмен информацией через огружающую среду посредством феромона.

Пусть муравей имеет следующие характеристики:
\begin{enumerate}[label=\arabic*)]
	\item зрение --- способность определить длину ребра;
	\item память --- способность запомнить пройденный маршрут;
	\item обоняние --- способность чуять феромон.
\end{enumerate}


Также введем целевую функцию \eqref{d_func}, характеризующую привлекательность ребра, определяемую благодаря зрению.

\begin{equation}
	\label{d_func}
	\eta_{ij} = 1 / D_{ij},
\end{equation}
где $D_{ij}$ — расстояние от текущего пункта $i$ до заданного пункта $j$.


Также понадобится формула вычисления вероятности перехода в заданную точку \eqref{posib}.

\begin{equation}
	\label{posib}
	p_{k,ij} = \begin{cases}
		\frac{\eta_{ij}^{\alpha}\cdot\tau_{ij}^{\beta}}{\sum_{q\notin J_k} \eta^\alpha_{iq}\cdot\tau^\beta_{iq}}, j \notin J_k \\
		0, j \in J_k
	\end{cases}
\end{equation}
где $a$ --- параметр влияния длины пути, $b$ --- параметр влияния феромона, $\tau_{ij}$ --- количество феромонов на ребре $ij$, $\eta_{ij}$ --- привлекательность ребра $ij$, $J_k$ --- список посещённых за текущий день городов.

После завершения движения всех муравьев (ночью, перед наступлением следующего дня), феромон обновляется по формуле \eqref{update_phero_1}.
\begin{equation}
	\label{update_phero_1}
	\tau_{ij}(t+1) = \tau_{ij}(t)\cdot(1-p) + \Delta \tau_{ij}(t).
\end{equation}
При этом
\begin{equation}
	\label{update_phero_2}
	\Delta \tau_{ij}(t) = \sum_{k=1}^N \Delta \tau^k_{ij}(t),
\end{equation}
где
\begin{equation}
	\label{update_phero_3}
	\Delta\tau^k_{ij}(t) = \begin{cases}
		Q/L_{k}, \textrm{ребро посещено муравьем $k$ в текущий день $t$,} \\
		0, \textrm{иначе}
	\end{cases}
\end{equation}

Поскольку вероятность перехода в заданную точку \ref{posib} не должна быть равна нулю, необходимо обеспечить неравенство $\tau_{ij} (t)$ нулю посредством введения дополительного минимально возможного значения феромона $\tau_{min}$ и в случае, если $\tau_{ij} (t+1)$ принимает значение, меньшее $\tau_{min}$, откатывать феромон до этой величины. 


Путь выбирается по следующей схеме.
\begin{enumerate}
	\item Каждый муравей имеет список запретов --- список уже посещенных городов (вершин графа).
	\item Муравьиное зрение отвечает за эвристическое желание посетить вершину.
	\item Муравьиное обоняние отвечает за ощущение феромона на определенном пути (ребре). При этом количество феромона на пути (ребре) в день $t$ обозначается как $\tau_{i, j} (t)$.
	\item После прохождения определенного ребра муравей откладывает на нем некотрое количество феромона, которое показывает оптимальность сделанного выбора, это количество вычисляется по формуле \eqref{update_phero_3}.
\end{enumerate}

\section{Требования к программе}
Выделим ряд требований к разрабатываемой программе:
\begin{itemize}[label=---]
	\item программа должна получать на вход матрицу смежности, для которой можно будет выбрать один из алгоритмов поиска оптимальных путей --- полным перебором или муравьиным алгоритмом;
	\item программа должна позволять пользователю определять коэффициенты и количество дней для муравьиного алгоритма;
	\item программа должна давать возможность получить минимальную сумму пути, а также сам путь, используя один из алгоритмов. Также должна присутствовать возможность провести тестирование по времени выполнения для разных размеров матриц.
\end{itemize}

\section*{Вывод}

В данном разделе была рассмотрена задача коммивояжёра, а также полный перебор для её решения и муравьиный алгоритм. Были представлены требования к разрабатываемому программному обеспечению.
